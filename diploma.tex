% В этом шаблоне используется класс spbau-diploma. Его можно найти и, если требуется,
% поправить в файле spbau-diploma.cls
\documentclass{spbau-diploma}
\begin{document}
% Год, город, название университета и факультета предопределены,
% но можно и поменять.
% Если англоязычная титульная страница не нужна, то ее можно просто удалить.
\filltitle{ru}{
    chair              = {Кафедра математических и информационных технологий},
    title              = {Генерация зависимых языков по спецификации пользователя},
    % Здесь указывается тип работы. Возможные значения:
    %   coursework - Курсовая работа
    %   diploma - Диплом специалиста
    %   master - Диплом магистра
    %   bachelor - Диплом бакалавра
    type               = {master},
    position           = {студента},
    group              = 604,
    author             = {Гарифуллин Шамиль Раифович},
    supervisorPosition = {аспирант},
    supervisor         = {Исаев В.\,И.},
    reviewerPosition   = {аспирант},
    reviewer           = {Подкопаев А.\,В.},
    chairHeadPosition  = {д.\,ф.-м.\,н., профессор},
    chairHead          = {Омельченко А.\,В.},
    % university = {САНКТ-ПЕТЕРБУРГСКИЙ АКАДЕМИЧЕСКИЙ УНИВЕРСИТЕТ},
    % faculty = {Центр высшего образования},
    % city = {Санкт-Петербург},
    % year             = {2013}
}
\filltitle{en}{
    chair              = {Department of Mathematics and Information Technology},
    title              = {Specification based generation of languages with dependent types},
    author             = {Shamil Garifullin},
    supervisorPosition = {PhD student},
    supervisor         = {Valeriy Isaev},
    reviewerPosition   = {PhD student},
    reviewer           = {Anton Podkopaev},
    chairHeadPosition  = {professor},
    chairHead          = {Alexander Omelchenko},
}
\maketitle
\tableofcontents
% У введения нет номера главы
\section*{Введение}
Языки программирования с зависимыми типами могут быть использованы для доказательств свойств кода программы. Также возможно ввести типы аналогичные сущностям области математики в которой мы хотим доказывать теоремы и просто писать термы, таким образом предъявляя доказательства утверждений. Плюс данного подхода заключается в том, что проверка доказательств перекладывается на тайпчекер.

Однако сами языки программирования, являясь достаточно общими, часто содержат слишком много конструкций для интересующей нас области, и приходится ограничивать язык на котором мы пишем. Также может быть такая ситуация, что конструкции, которыми мы хотим пользоватсья, не существуют в языке программирования. Поэтому если мы хотим переложить проверку наших высказываний на тайпчекер приходится писать свой язык программирования и уже в нем доказывать утверждения.

Решение описанной проблемы - генерация зависимых языков по спецификации конструкций, которые мы хотим от нашего языка является темой данной работы.

\section{Постановка задачи}

Целью данной работы является дизайн и имплементация языка для спецификации языков программирования с зависимыми типами. Ключевые задачи которые решает работа:
\begin{itemize}
  \item Сужение множества возможных спецификаций зависимых языков, для того чтобы была возможна генерация тайпчекера
  \item Реализация генерации структур данных представления языка и функций манипуляции этими структурами.
  \item Реализация генерации функций приведения термов специфицированного языка в нормальную форму и проверки типов.
\end{itemize}

\section{Зависимые языки} \label{deptypes_intro}
Языки с зависимыми типами позволяют типам зависеть от термов, то есть мы, например, можем иметь тип списков фиксированной длины. Это позволяет нам описывать ограничения налагаемые на использование функций, которые мы пишем.

Одной из наиболее частых ошибок при программировнии на языке вида Haskell является взятие первого элемента пустого списка.

\begin{lstlisting}[frame=single]
head :: [a] -> a
head (x:_) = x
head [] = error "No head!"
\end{lstlisting}

Которая легко решается, если мы можем иметь термы языка в типе.

\begin{lstlisting}[frame=single]
head :: {n : N} -> Vec a (suc n) -> a
head (x:_) = x
\end{lstlisting}

Здесь тип явно специфицирует, что функция не принимает термы типа 'Vec a 0'

Этот способ обобщается, и можно доказывать корректность работы алгоритмов, например функции filter в Приложении~\ref{sort_proof}.

\subsection{Проверка типов в зависимых языках}\label{typecheck}
Рассмотрим пример:

\begin{center}
\AxiomC{$\Gamma, x : S \vdash T\ type $}
\AxiomC{$\Gamma, \vdash f : pi(S, T) $}
\AxiomC{$\Gamma \vdash t : S $}
\TrinaryInfC{$\Gamma \vdash app(T, f, t) : T[x:=t]$}
\DisplayProof
\end{center}

Если считать, что заключение правила вывода, то проверка типов в любом языке происходит так: мы имеем некоторые аргументы внутри примитива, которые мы используем для составления узлов-потомков (предпосылок).

На этих узлах вызываем функцию вывода типов в возможно расширенном контексте\footnote{Конечно, мы должны для каждого расширения контекста проверять его корректность.} рекурсивно. Если потомки составлены корректно, то получаем некие типы, которые можем использовать в проверке равенств в предпосылках и возрате типа примитива.

В зависимых языках все точно так же, однако проверка на равенство должна происходить после нормализации термов. Нормализацию мы применяем только после того как убедимся, что термы корректно составлены. То есть имеем, что нормализация тесно связана с проверкой типов. Более того проверка типов невозможна без нормализации термов.

Действительно, чтобы понять что $2 + 3 = 5$, мы должны провести вычисления и убедиться в этом.

\subsection{Индексы де Брейна}\label{de_brujin}
При реализации функциональных языков одной из самых сложных частей является написание подстановок. Большинство проблем и ошибок в реализации тоже связано с ней.

Одной из таких проблем является сравнение альфа-эквивалентных термов. Альфа-эквивалентными называются термы, которые отличаются только в именовании связанных переменных. Например, следующие три терма альфа-эквивалентны:

\begin{lstlisting}
lamb x y → y (x z)
lamb y x → x (y z)
lamb a b → b (a z)
\end{lstlisting}

Понятно, что мы сталкиваемся с проблемами при использовании переменных в виде строк, например первый терм сверху выглядел бы как \lstinline{[Lam "x" (Lam "y" (App "y" (App "x" "z")))]}. И проверка равенства этого терма терму \lstinline{[Lam "y" (Lam "x" (App "x" (App "y" "z")))]} занятие, склонное к ошибкам.

Другой проблемой такого представления термов является избегание захвата переменных при подстановке. Положим, мы подставляем первый терм ниже в переменную "z" во втором.
\begin{lstlisting}
lamb x → y
lamb y → z
lamb y → lamb x → y = lamb y x → y
\end{lstlisting}

Очевидно, что подставлять в переменную так наивно нельзя, так как "y" стала связанной, хотя не была таковой в первоначальном терме.

Ключевым замечанием является то, что переменные в функциональных языках являются "указателями" на место их связывания --- этаким индексом в контекст --- и не несут никакой дополнительной информации.

Результат использования этого наблюдения называется индексами де Брейна. А именно: для каждой связанной переменной мы просто пишем расстояние от неё до ближайшего связывания.

Если переписать термы с альфа эквивалентностью выше, то для всех трех термов получим \lstinline{[lamb lamb → 1 (2 z)]}, и проверка на альфа-эквивалентность превращается в проверку на равенство.

Также решается проблема избегания захвата переменных, а именно:
\begin{lstlisting}
lamb → y
lamb → z
lamb → lamb → y = lamb lamb → y
\end{lstlisting}

Как видно "y" остался свободным.

Это представление значительно лучше удовлетворяет нашим требованиям разработчика языков. Мы перешли от
\lstinline{[Lam "y" (Lam "x" (App "x" (App "y" "z")))]} к \lstinline{[Lam (Lam (App 1 (App 2 "z")))]}.

Однако общей проблемой обоих представлений является нетипизированность переменных --- никто не контролирует построение термов вида \lstinline{[Lam (Lam (App 123 (App 23 "z")))]}. Решение этой проблемы описано в секции~\ref{de_brujin_impl}.









%--

\section{Обзор аналогов}
Построение языков программирования с зависимыми типами по спецификации является задачей достаточно специфичной. Ниже перечислены инструменты, применяемые в похожих ситуациях (изучение формальных систем, языков программирования и их реализация).

\subsection{BNFC}
Чем-то похожим средством разработки является BNFC\cite{bnfc}. Позволяет генерировать фронтенд компилятора по аннотированной грамматике языка в форме Бэкуса-Наура\cite{lbnf}.

Генерирует лексер, парсер и преттипринтер языка. Также создает АСТ и заготовку для написания редукций (просто один большой case).

\subsection{PLT/Redex}
PLT/Redex\cite{plt:redex} --- EDSL Racket созданный для спецификации и изучения операционных семантик. Может быть использован для спецификации языков программирования, в том числе и с зависимыми типами.

Позволяет рандомизированно тестировать цикличность редукций или иные свойства языка. Также позволяет визуализировать редукции.

Однако спецификация языков с зависимыми типами занимает столько же усилий, как если бы пользователь писал реализацию языка в Haskell\cite{plt:ex}. Большую сложность составляют подстановки --- проблема, обойденная в данной работе с использованием представления в виде индексов де Брейна.

\subsection{Twelf}
Twelf\cite{twelf} является реализацией LF\cite{Pfenning2002}. Используется для спецификации и доказательств свойств логик и языков программирования.

Для спецификации задаются высказывания языка (используется принцип "высказывания в качестве типов"\cite{harper:1993}) и некоторые операции над ним в виде отнощений. Затем доказываются  $\forall\Sigma$-свойства данного языка.

Таким образом, в Twelf можно доказывать свойства спецификаций языков или приводить спецификации к форме, в которой выполняются интересующие нас свойства. Затем можно использовать программу описанную в данной работе для имплементации языка.






%--

\section{Определение языка спецификаций}\label{lang_spec}

Вдохновением данной работы послужила статьи~\cite{Palmgren} и~\cite{isaev}. Поэтому сам язык спецификации выглядит как язык описания алгебраических теорий.\footnote{А именно: помимо правил вывода у нас есть сорта и функциональные символы.

Каждая конструкция в языке --- это функциональный символ в логике, а правила вывода и редукции --- это аксиомы.
Правила вывода говорят когда некоторый функциональный символ определен.

Все функциональные символы являются частичными функциями, поэтому это существенно алгебраические теории, а не просто алгебраические.}

Начнем с примера описания языка с зависимыми типами~(рис.\ref{lpi})~\cite[Глава~2.1]{book:pierce}

\begin{figure}
    \centering
	\includegraphics[scale=0.35]{img/lp.png}
	\caption{Язык с лямбдой и $\Pi$-типами }
	\label{lpi}
\end{figure}

В данном нас явно выделяются три сорта (можно думать о сортах как о метатипах): кайнды, термы и типы (правила связанные с кайндами и само их описание опущены для простоты).

Также явно выделяются примитивы языка\footnote{В дальнейшем мы называем их функциональными символами.}:
абстракция, пи-типы (стрелки в языках без зависимых типов) и аппликация. Легко заметить, что во всех яхыках присутствуют подстановка, контексты, символ ':' означающий, что тип терма слева есть терм справа, и связывание переменных.

Правила вида T-Conv и T-Var всегда верны в зависимых языках, поэтому у нас они есть по умолчанию. Также подразумевается рефлексивность, симметричность, транзитивность и конгруэнтность равенства.

Если принять во внимания все наблюдения выше то так этот язык будет выглядеть в нашем языке спецификации\footnote{Важно понимать, что запись $\_ \vdash$ не означает, что контекст пуст, если слева ничего не написано, это эквивалентно записи $\Gamma \vdash$.}:

\begin{lstlisting}[frame=single]
DependentSorts:
  tm, ty
FunctionalSymbols:
  lam: (ty, 0)*(tm, 1) -> tm
  app: (tm, 0)*(tm, 0)*(ty, 1) -> tm
  pi : (ty, 0)*(ty, 1) -> ty
Axioms:
  K-Pi =
    forall T1 : ty, x.T2 : ty
      x : T1 |- T2 def |--- |- pi(T1, x.T2) def

  TAbs =
    forall S : ty, x.T : ty, x.t : tm
      x : S |- t : T |--- |- lam(S, x.t) : pi(S, x.T)
  TApp =
    forall t1 : tm, t2 : tm, S : ty, x.T : ty
            |- t1 : pi(S, x.T),
            |- t2 : S,
      x : S |- T def
      |--------------------------
      |- app(t1, t2, x.T) : T[x:=t2]
Reductions:
  Beta =
    forall x.b : tm, A : ty, a : tm, z.T : ty
       |--- |- app(lam(A, x.b), a, z.T) => b[x:=a] : T[z:=a]

\end{lstlisting}

Типизирование метапеременных позволяет проверять правильность применения функциональных символов и наличие нужных переменных в контексте. Именованные переменные служат для определения порядка переменных в контексте и не несут какой-то дополнительной информации.

Также в язык была добавлена проверка на c-стабильность --- можно помечать аксиомы типами, тогда аксиома применима, только если все переменные входящие в терм являются представителями этих типов\footnote{Если список типов пуст, то производится проверка на отсутствие свободных переменных.}.

\subsection{Ограничения на спецификации, налагаемые языком}

Если рассматривать спецификации как произвольные существенно алгебраические теории, то пользователь может написать спецификацию, для которой мы не сможем сгенерировать тайпчекер. Поэтому вводятся следующие ограничения на спецификации языков:

\begin{enumerate}
\item Запрещено равенство в заключении аксиом для определенности каждого шага в проверке типов определяемого языка Это связано с тем, что, если мы видим равенство, не ясно в какую сторону идти при редуцировании.

\item Если в заключении аксиомы написан функциональный символ возвращающий сорт термов, он обязан также иметь тип (нельзя просто написать $ \vdash f(\ldots) def$). Так как становится неясно какой тип возвращать при выводе типов.

\item Определения функциональных символов всегда одно, иначе появляется недетерминированность в проверке типов. Не играет особой роли, так как в данном случае можно сделать недетерминированность в проверке.

\item Подстановки разрешены только в метапеременные --- в принципе, это слабое ограничение, которое облегчает жизнь при реализации, не ограничивая пользователя.

\item \label{tm:Meta} Все метапеременные используемые в предпосылках должны либо присутствовать в метапеременных заключения или же должны быть типами какой-либо предпосылки. Иначе нужно считать, что это верно для любого представителя сорта метапеременной.

\item Если в функциональном символе встречаются метапеременные с контекстами $x_1 \ldots x_k . T$ должна существовать предпосылка вида $x_1 : S_1 \ldots x_k : S_k  \vdash T$. Это сделано для того чтобы не передавать типы контекстов метапеременных функционального символа явно.

\item Если метапеременная является типом предпосылки и не встречается в аргументах функционального символа, то она может использоваться только справа от двоеточия. Таким образом избегаются ситуации связанные с порядком проверки предпосылок языка. А именно: если у нас есть $x : S \vdash t : T,\ x:T \vdash r : S$, то нужно строить граф зависимостей для предпосылок и использовать порядок полученный в результате его топологической сортировки в генерации кода. (Аналогично с~\ref{toposort}).

\item \label{order:Meta} Все переменные контекстов определения метапеременных могут использовать только метапеременные левее внутри функционального символа в заключении --- это связано с тем, что иначе могут возникнуть циклы в определениях метапеременных: S тип с аргументом типа R, R тип с аргументом типа S, S тип с аргументом типа R...

\item Из-за ослабления условия на метапеременные в Пункте~\ref{tm:Meta}, порядок метапеременных неочевиден. Решение данной проблемы и~(\ref{order:Meta}) описано в Секции~\ref{toposort}.

\item Редукции не учитывают предпосылок при приведении в нормальную форму --- предполагается что они не конфликтуют с аксиомами и проверки в аксиомах достаточно.

\item В редукциях все метапеременные справа от '=>' должны встречаться и слева от него. Иначе непонятно откуда их взять.

\item Подстановка запрещена слева от '=>'.

\item Все редукции всегда стабильны.

\item Все аргументы в функциональный символ в заключении аксиомы должны быть метапеременными --- случай с не только метапеременными требует дальнейшего исследования. Ещё и с теми же аргументами, что и в forall (не расширенный котекст, не существенно).

\item В заключении контекст не должен быть расширен --- это ограничение связано с тем, что иначе смысл аксиомы становится странным. А именно: функциональный символ применим только при введении переменных в контекст.

\end{enumerate}

Также у нашего языка есть ограничения, налагаемые существенно-алгебраическими теориями:
\begin{itemize}
\item Все используемые метапеременные должны иметь аннотацию (сорт), то есть присутствовать в секции forall аксиомы/редукции.
\item Мы явно специфицируем все сорты, которые используем.
\end{itemize}

\subsection{Проверки корректности спецификации языка}

Все ограничения выше проверяются при обработке спецификации языка.

Также тривиальными проверками, осуществляемыми после парсинга языка, являются:
\begin{itemize}
\item Все метапеременные, используемые в правилах вывода/редукциях находятся в контексте включающем их контекст описанный в секции forall.
\item Проверка того, что сорты используемых выражений совпадают с сортами аргументов функциональных символов.
\item Подстановка осуществляется в переменные, которые есть в свободном виде в метапеременной.
\item Контексты метапеременных содержат все их метапеременные.
\item Все функциональные символы имеют ассоциированное правило вывода.
\end{itemize}

\section{Реализация}


\subsection{Парсер генераторы}

\subsection{Индексы де Брейна и их проблемы(задачки с индексами)}

\subsection{Проверка типов}

\subsection{сама генерация кода - просто описать exts + структуру}

\subsection{Упорядочивание переменных в функциональном символе}\label{toposort}






%--


% У заключения нет номера главы
\section*{Заключение}

\bibliographystyle{ugost2008ls}
\bibliography{diploma}

\appendix
\section*{Приложения}
\addcontentsline{toc}{section}{Приложения}
\renewcommand{\thesubsection}{\Alph{subsection}}

\subsection{Доказательство корректности функции sort}\label{sort_proof}

Ниже показан пример доказательства того, что функция filter выдает подсписок исходного списка.
Код написан на Agda\cite{agda}

\begin{lstlisting}
-- Определяем предикат принадлежности элемента списку.
data _∈_ {A : Set} (a : A) : List A → Set where
  here : (xs : List A) -> a ∈ (a ∷ xs)
  there : (x : A) (xs : List A) -> a ∈ xs -> a ∈ (x ∷ xs)

-- Определяем предикат xs ⊆ ys, означающий "список xs является подсписком ys".
data _⊆_ {A : Set} : List A -> List A → Set where
    nil : [] ⊆ []
    larger : {y : A} {xs ys : List A} -> xs ⊆ ys -> xs ⊆ (y ∷ ys)
    cons  : {x : A} {xs ys : List A} -> xs ⊆ ys -> (x ∷ xs) ⊆ (x ∷ ys)

-- Докажем, что filter xs ⊆ xs для любого списка xs.
filter' : {A : Set} -> (A → Bool) → List A → List A
filter' p [] = []
filter' p (x ∷ xs) = if p x then x ∷ filter' p xs else filter' p xs

filterLess : {A : Set} -> (p : A -> Bool) -> (xs : List A) -> filter' p xs ⊆ xs
filterLess p [] = nil
filterLess p (x ∷ xs) with p x
filterLess p (x ∷ xs) | false = larger (filterLess p xs)
filterLess p (x ∷ xs) | true = cons (filterLess p xs)

\end{lstlisting}

\subsection{Индексы де Брейна}\label{de_brujin}
 % Описание самих индексов и альфа эквивалентности и подстановки


%--


\end{document}
