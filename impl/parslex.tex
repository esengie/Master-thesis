\subsection{Генераторы синтаксических анализаторов} \label{pars_generators}
В ходе всей работы использовались генераторы лексических и синтаксических анализаторов alex\cite{alex} и happy\cite{happy}.

Решение использовать именно генераторы синтаксических анализаторов, а не комбинаторы синтаксических анализаторов\cite{parsec} или другие методы синтаксического анализа было обуcловлено тем, что прогнозировались частые изменения грамматики вместе с эволюцией языка.

\begin{lstlisting}[caption={Часть спецификации синтаксического анализатора},captionpos=b, frame=single, label={lst_happy}]
Axiom   :   Header '=' '\t' Forall '\t'
            Premise '|---' JudgementNoEq '/t' '/t'
              { Axiom (snd $1) (fst $1) $4 $6 $8 }
        |   Header '=' '\t'
            Premise '|---' JudgementNoEq '/t'
              { Axiom (snd $1) (fst $1) [] $4 $6 }
\end{lstlisting}

Все изменения связанные с грамматикой языка проводились на уровне спецификации АСД. По вставке~\ref{lst_happy} можно судить, что изменения грамматики непосредственно ложатся на изменения спецификации синтаксического анализатора. При изменении грамматики мы меняем их в спецификации напрямую, в крайних случаях ещё приходится менять представление АСД языка, возвращаемого анализатором.

\hfill

Стоить заметить, что в языке спецификации отступы значительны. Это известная проблема реализации лексического/синтаксического анализа --- так как такая грамматика не является контекстно-свободной. В работе была решена с помощью монадического лексического анализатора, который преобразовывал отступы в аналог открывающих и закрывающих скобок.

Сам синтаксический анализатор очень простой и однопроходный. Поэтому все, что выглядит как переменная, считается переменной. Для того, чтобы отделить метапеременные и конструкции нулевой арности применяется второй проход по структуре, выдаваемой первым проходом.

% \begin{minipage}{\linewidth}
% \begin{lstlisting}[caption={АСД языка спецификации},captionpos=b,frame=single, label={lst_langspec}]
% data LangSpec = LangSpec {
%   stabilities     :: Stab
% , depSortNames    :: [SortName]
% , simpleSortNames :: [SortName]
% -- These are language constructs
% , funSyms         :: [FunctionalSymbol]
% , axioms          :: [Axiom]
% , reductions      :: [Reduction]
% }
%
% -- These
% data Axiom = Axiom {
%   name       :: Name
% , stab       :: Stab
% , forallVars :: [(MetaVar, Sort)]
% , premise    :: [Judgement]
% , conclusion :: Judgement
% }
%
% data Judgement =
%   Statement {
%   jContext   :: [(VarName, Term)]
% , jTerm :: Term
% , jType :: Maybe Term    -- def as maybe
% } |
%   Equality {
%   jContext   :: [(VarName, Term)]
% , jLeft  :: Term
% , jRight  :: Term
% , jType :: Maybe Term -- equality t1 = t2 : Maybe t3
% }
%
% data Term = Var VarName
%           | Meta MetaVar
%           | FunApp Name [(Ctx, Term)]
%           | Subst Term VarName Term
%     deriving (Eq)
% \end{lstlisting}
% \end{minipage}
%
% АСД языка спецификации, после синтаксического анализа имеет представление, частично написанное во вставке~\ref{lst_langspec}. По представлению можно судить, что никаких взяимосвязей между конструкциями
%
% Синтаксический анализатор выдает АСД языка спецификации, которое идет на вход алгоритму проверки спецификации\footnote{Можно сравнить с гораздо более структурированной структурой (см. вставку~\ref{SymTab}), выдаваемой алгоритмом на выходе.}.

%%
