\subsection{Модуль проверки корректности спецификации}\label{sortcheck}
Вся проверка корректности проходит внутри монады SortCheckM, которая является стэком монад StateT и Either.

Монада Either используется для обработки ошибок. А монада State нужна, так как в ходе работы алгоритма постепенно заполняется таблица определений языка спецификации.

\begin{lstlisting}[caption={Структура заполняемая модулем проверки спецификации},captionpos=b,frame=single, label={SymTab}]
data SymbolTable = SymbolTable {
  depSorts      :: Set AST.SortName
, simpleSorts   :: Set AST.SortName
, funSyms       :: Map AST.Name AST.FunctionalSymbol
, axioms        :: Map AST.Name Axiom
, reductions    :: Map AST.Name Reduction
-- rules of constructs
, iSymAxiomMap  :: Map AST.Name AST.Name
}
\end{lstlisting}

Эта структура содержит:
\begin{itemize}
\item Множества всех зависимых и независимых сортов выражений
\item Таблицу определений всех конструкций, которые содержат их арности и сорта
\item Таблицы правил вывода и редукции, содержащие их определения
\item Таблицу соответствия конструкций их правилу вывода
\end{itemize}

Изначально заполняются множества зависимых и независимых сортов. Затем происходит проверка и заполнение определения конструкций.

Сами правила вывода и редукции, ввиду однопроходности синтаксического анализатора (упоминалось в Разделе~\ref{pars_generators}), могут быть заполнены изначально некорректно. Все 0-арные конструкции языка и все метапеременные синтаксическим анализатором распознаются как переменные. Это поправляется на этапе рекурсивного обхода переменных. При встрече переменной сперва просматривается таблица конструкций, затем метапеременных правила вывода/редукции. Если ни там, ни в другом месте ничего не находится считается, что это переменная и проверяется на отсутствие перекрытия других переменных из контекста.

Затем проводятся проверки описанные в Разделе~\ref{constraints}. Эти проверки достаточно очевидно переводятся в код, поэтому описывать их здесь мы не сочли нужным.




















%%%
