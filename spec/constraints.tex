\subsection{Ограничения на спецификации, налагаемые языком}\label{constraints}
Если не налагать никаких ограничений на спецификации, то пользователь может написать спецификацию, для которой мы не сможем сгенерировать функцию проверки типов. Поэтому вводятся следующие ограничения на спецификации языков:

\begin{enumerate}
\item Запрещено перекрытие переменных, которые уже есть в контексте. Это чисто стилистическое ограничение, которое не вносит никаких ограничений на сами языки.

\item \label{directed_reducts} Запрещено равенство в заключении правил вывода. Это сделано для определенности каждого шага в проверке типов определяемого языка и связано с тем, что, если мы видим равенство, не ясно в какую сторону идти при редуцировании. Поэтому мы заставляем пользователя пользоваться редукциями, а не равенствами в заключении. Поэтому все редукции направленные.

\item Все аргументы в конструкцию языка в заключении правила вывода должны быть метапеременными --- случай содержащий не только метапеременные требует дальнейшего исследования. Ещё и с теми же контекстами, что и в forall --- если бы контексты были шире, то сразу же приходилось бы проверять их на возможность удаления лишних переменных.

\item В заключении контекст не должен быть расширен --- это ограничение связано с тем, что иначе смысл правила вывода становится странным. А именно: конструкция применима только при введении переменных в контекст.

\item Если в заключении правила вывода написана конструкция возвращающая сорт термов, она обязана быть проаннотирована типом (нельзя просто написать $ \vdash f(\ldots) def$). Так как иначе становится неясно какой тип возвращать при выводе типа данной конструкции.

\item \label{one_per_fun} Правило ввывода для каждой конструкции всегда одно, иначе появляется недетерминированность в проверке типов. Не играет особой роли, так как в данной ситуации возможно сделать недетерминированность при проверке типов. Однако в ходе эксплуатации не возникало нужды в обратном. Понадобилось бы более тщательное обдумывание последствий отсутствия данного ограничения. В будущем возможны изменения.

\item Подстановки разрешены только в метапеременные --- в принципе, это слабое ограничение, которое облегчает жизнь при реализации, не ограничивая пользователя. Нам, как разработчикам, не нужно ещё и на уровне мета-языка заботиться о подстановках.

\item \label{tm:Meta} Все метапеременные, используемые в предпосылках, должны либо присутствовать в метапеременных заключения, или же должны быть типами какой-либо предпосылки. Иначе попросту не ясно откуда брать эти метапеременные при проверке типов.

\item Если в конструкции языка встречаются метапеременные с контекстами $x_1 \ldots x_k . T$, то должна существовать предпосылка вида $x_1 : S_1 \ldots x_k : S_k  \vdash T$ в правиле вывода конструкции. Это сделано для того, чтобы не передавать типы контекстов метапеременных конструкции явно, а выводить их из таких условий. Предпосылки такого вида мы называем \textit{определениями метапеременной}.

\item Если метапеременная является типом предпосылки и не встречается в аргументах функционального символа, то она может использоваться только справа от двоеточия. Таким образом избегаются ситуации связанные с порядком проверки предпосылок языка. А именно: если у нас есть $x : S \vdash t : T,\ x:T \vdash r : S$, то нужно строить граф зависимостей для предпосылок и использовать порядок полученный в результате его топологической сортировки в генерации кода. (Аналогично с~\ref{toposort}).

\item \label{order:Meta} Все переменные контекстов определения метапеременных (то есть предпосылки в правиле вывода конструкции) могут использовать только метапеременные левее внутри конструкции в заключении --- это связано с тем, что иначе могут возникнуть циклы в определениях метапеременных: S тип с аргументом типа R, R тип с аргументом типа S, S тип с аргументом типа R и т.д.

\item Из-за ослабления условия на метапеременные в Пункте~\ref{tm:Meta}, порядок метапеременных неочевиден. Решение данной проблемы и~(\ref{order:Meta}) описано в Разделе~\ref{toposort}.

\item Редукции не учитывают предпосылок при приведении в нормальную форму --- предполагается что они не конфликтуют с правилами вывода и проверок корректности термов в правилах вывода достаточно.

\item В редукциях все метапеременные справа от `\lstinline{=>}' должны встречаться и слева от него. Иначе непонятно откуда взять эти метаперемнные при формировании правой части редукции.

\item Подстановка запрещена слева от `\lstinline{=>}'. Это сделано для возможности сопоставления с образцом при генерации функции приведения в нормальную форму. Также не очень понятно, как восстанавливать исходную метапеременную, до подстановки.

\item Все редукции всегда стабильны. Иначе требует дальнейшего исследования, так как появится требование передачи контекста в функцию нормализации.

\end{enumerate}

\subsection{Проверки корректности спецификации языка}

Все ограничения выше проверяются при обработке спецификации языка.

Также тривиальными проверками, осуществляемыми после парсинга языка, являются:
\begin{itemize}
\item Все метапеременные, используемые в правилах вывода/редукциях находятся в контексте, включающем их контекст, описанный в секции forall.
\item Проверка того, что сорты используемых выражений совпадают с сортами аргументов конструкций.
\item Подстановка осуществляется в переменные, которые есть в свободном виде в метапеременной.
\item Контексты метапеременных содержат все их переменные.
\item Все конструкции языка имеют ассоциированное правило вывода.
\end{itemize}
