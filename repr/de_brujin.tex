\subsection{Традиционные индексы де Брейна}\label{de_brujin}
В работе подразумевается реализация языков программирования через описание АСД на Haskell. При реализации функциональных языков одной из первых проблем встающих перед программистом является выбор представления АСД. Также нужно описывать абстракцию терма по переменной, инстанциирование переменной терма каким-то другим термом и проверку термов на равенство, и многие задачи и ошибки в реализации связаны именно с этими операциями.

Одной из задач представления термов является сравнение $\alpha$-эквивалентных термов. \textit{$\alpha$-эквивалентными} называются термы, которые отличаются только в именовании связанных переменных. Например, следующие три терма $\alpha$-эквивалентны:

\begin{lstlisting}
lamb x y → y (x z)
lamb y x → x (y z)
lamb a b → b (a z)
\end{lstlisting}

Одним из возможных способов представления термов является представление переменных в виде строк. С использованием такого подхода первый приведенный выше терм записывается в виде \lstinline{[Lam ``x'' (Lam ``y'' (App ``y'' (App ``x'' ``z'')))]}. Проверка равенства этого терма второму терму \lstinline{[Lam ``y'' (Lam ``x'' (App ``x'' (App ``y'' ``z'')))]} не тривиальна.

Другой проблемой такого представления термов является захват свободных переменных при подстановке. Предположим, мы подставляем первый терм ниже в переменную ``z'' во втором.
\begin{lstlisting}
lamb x → y
lamb y → z
lamb y → lamb x → y = lamb y x → y
\end{lstlisting}

Очевидно, что подставлять в переменную так наивно нельзя, так как ``y'' стала связанной, хотя не была таковой в первоначальном терме.

Ключевым замечанием является то, что переменные в функциональных языках являются ``указателями'' на место их связывания --- этаким индексом в контекст --- и не несут никакой дополнительной информации.

Результат использования этого наблюдения называется индексами де Брейна. А именно: для каждой связанной переменной мы просто пишем расстояние от неё до ближайшего связывания.

Если переписать термы с альфа эквивалентностью выше, то для всех трех термов получим \lstinline{[lamb lamb → 1 (2 z)]}, и проверка на альфа-эквивалентность превращается в проверку на равенство.

Также решается проблема захвата свободных переменных, а именно:
\begin{lstlisting}
lamb → y
lamb → z
lamb → lamb → y = lamb lamb → y
\end{lstlisting}

Как видно ``y'' остался свободным.

Это представление значительно лучше удовлетворяет нашим требованиям разработчика языков. Мы перешли от
\lstinline{[Lam ``y'' (Lam ``x'' (App ``x'' (App ``y'' ``z'')))]} к \lstinline{[Lam (Lam (App 1 (App 2 ``z'')))]}.

Однако общей проблемой обоих представлений является нетипизированность переменных --- никто не контролирует построение термов вида \lstinline{[Lam (Lam (App 123 (App 23 ``z'')))]}. Решение этой проблемы описано в разделе~\ref{de_brujin_impl}.
