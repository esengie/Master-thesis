\subsubsection{Индексы де Брейна на уровне типов}\label{de_brujin_impl}
В нашем описании индексов де Брейна в Cекции~\ref{de_brujin} мы упомянули, что наивное их использование склонно к ошибкам и не использует систему типов Haskell.

Эту проблему можно решить с помощью полиморфной рекурсии\cite{Bird:Pat}. По сути, каждый раз когда мы абстрагируемся по переменной в представлении де Брейна, мы добавляем единицу ко всем свободным переменным внутри выражения. Ключевым наблюдением является то, что мы можем добавлять единицу оборачивая выражение в Maybe. Например:

\begin{lstlisting}[frame=single]
data Term a
  = Var a
  | App (Term a) (Term a)
  | Lam (Term (Maybe a))
\end{lstlisting}

Стоит сказать, что, если добиться некоторой абстрактности представления, это позволит нам генерировать меньше кода. Идея состоит в определении для представления выражений представителя класса Monad, смыслом операции bind будет применение функций к переменным. Через неё можно выразить подстановку, при этом ниже будет показано, каким образом можно выделить связывания в свой модуль так, чтобы выражения заботились только о подстановке в переменные, а связывания обрабатывались бы в коде этого модуля.

Поэтому метод выше не очень удобен при кодогенерации, так как функции подстановки и абстракции будут сильно зависеть от определения Term и нам придется генерировать много кода, специфичного для каждого представления (полный пример кода для данного представления можно увидеть на \url{github.com/esengie/cath_lec/blob/master/lec7/tasks7.hs}).

В той же статье\cite{Bird:Pat} предложен способ превращения этого паттерна программирования в трансформер монад. В последующем коде Maybe заменен на Var, в соответствии со своей семантикой, но отличие только в названии. Все представители классов у Var работают так же, как и у Maybe (Alternative, Functor, Monad и проч.). Также можно заметить, что Scope есть трансформер монад MaybeT (прочитать, как именно это работает для Maybe и MaybeT, можно в книге\cite{moronuki}).

\begin{lstlisting}[frame=single]
data Var a = B | F a
newtype Scope f a = Scope { fromScope :: f (Var a) }

toScope :: f (Var a) -> Scope f a
toScope = Scope

instance Monad f => Monad (Scope f) where
  return = Scope . return . F
  Scope m >>= f = Scope $ m >>= varAppWithDefault (return B) (fromScope . f)

instance MonadTrans Scope where
  lift = Scope . liftM F
\end{lstlisting}

Теперь мы можем написать общие функции абстрагирования по переменной и подстановки в самую внешнюю связанную переменную выражения.
\begin{lstlisting}[frame=single]
abstract :: (Functor f, Eq a) => a -> f a -> Scope f a
abstract x xs = Scope (fmap go xs) where
  go y = y <$ guard (x /= y)

instantiate :: Monad f => f a -> Scope f a -> f a
instantiate x (Scope xs) = xs >>= go where
  go B = x
  go (F y) = return y
\end{lstlisting}

Функция abstract при совпадении с абстрагируемой переменной, пользуясь определением Alternative Var, возвращает B -- что означает связанную переменную. Иначе она, пользуясь определением Applicative Var, возвращает pure, что есть F, то есть повышает индекс другой переменной внутри выражения, так как появилось ещё одно связывание между переменной и местов её связывания.

Функция instantiate просто подставляет в B переменную (так как она наиболее внешняя), иначе понижает индекс переменной, так как одно связывание между ней и местов её связывания исчезло.

Теперь при генерации кода нам всего лишь понадобится определить гораздо более простую монаду подстановок для АСД выражений, а со связываниями разбирается наш трансформер Scope. Представление и представитель класса Monad, которое нужно генерировать, выглядят теперь так:

\begin{lstlisting}[frame=single]
data Term a
  = Var a
  | App (Term a) (Term a) (Scope Term a)
  | Lam (Term a) (Scope Term a)

instance Monad Term where
  Var v1 >>= f = f v1
  App v1 v2 >>= f = App (v1 >>= f) (v2 >>= f)
  Lam v1 v2 >>= f = Lam (v1 >>= f) (v2 >>>= f)

(>>>=) :: (Monad f) => Scope f a -> (a -> f b) -> Scope f b
m >>>= f = m >>= lift . f
\end{lstlisting}

Где оператор \lstinline{(>>>=)} работает как обычная композиция с lift нашего трансформера Scope.

Также стоит описать роль функций toScope и fromScope, в виду их частого использования в реализации. Они предназначены для обертки выражений в тип данных Scope и выноса их из него (для удобства введены несколько функций $fromScope_i$ и $toScope_i$, которые являются композициями исходных). Эти функции служат для вхождения под связывания и выхода из них. Тип `$Term\ a \rightarrow Term\ (Var\ a)$' --- означает, что выражение переходит в тип с контекстом большим на единицу. Например при наличии у нас метапеременной $\Gamma \vdash (x y z).T$, fromScope2 переведёт её вот в такой вид $\Gamma, x, y \vdash z.T$ --- важным моментом здесь является, что контекст $\Gamma$ должен быть расширен соответcтвующим образом. Для этого есть функция consCtx, которая имеет тип `$Ctx\ a \rightarrow Ctx\ (Var\ a)$', которая занимается переносом связывания в контекст. Таким образом тип выражения следит за количеством свободных переменных в нём --- переменные просто переходят в контекст при вхождении под связывания этих переменных.

\hfill

Стоить отметить, что этот метод c некоторыми оптимизациями использован в библиотеке bound\cite{bound}. В подходе выше нужно каждую переменную оборачивать в F индивидуально, в библиотеке использован способ оборачивания поддеревьев АСД выражений.

Однако в нашем случае эта оптимизация неприменима, так как ещё одной задачей, которую должно решать представление --- является возможность сопоставления с образцом. Например, если у нас есть искусственная редукция вида $app(\lambda[A, x.\lambda(B, y.t)), r] => t[x:=r]$, в функции приведения в нормальную форму происходит сопоставление с образцом, а именно: левая часть функции нормализации выглядит как-то так `\lstinline{nf (App (Lam _ (Scope (Lam _ t))) r) = ...}' --- основная важность этого примера в том, что наше представление позволяет заходить под связывание при сопоставлении с образцом, чего не позволяет сделать библиотека bound.
