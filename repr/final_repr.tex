\subsubsection{Представление выражений}\label{final_repr}

В итоге, было выбрано представление описанное в Разделе~\ref{de_brujin_impl}.

Осталось только описать, как можно реализовать равенство для выражений вида `\lstinline{Term a}'. Как видим, вид этого типа в Haskell $* \rightarrow *$, и для него можно определить только представителей высших классов\cite{prel_extras} Eq1 и Show1.

Так как равенство выражений просто структурное, его возможно сгенерировать с помощью Template Haskell\cite{TH}. Представители классов Eq и Show получаются с помощью механизма DeriveEq1, DeriveShow1\cite{deriveCompat}.

Затем мы просто пишем определения представителей, независящие от представления (см. вставку~\ref{lst_inst_eq})

\begin{lstlisting}[caption={Определение представителей классов Eq и Show для представления АСД}, captionpos=b, frame=single, float,floatplacement=H, label = {lst_inst_eq}]
instance Eq a => Eq (Term a) where (==) = eq1
instance Show a => Show (Term a) where showsPrec = showsPrec1
\end{lstlisting}

Таким образом использование полиморфной рекурсии для выражения индексов де Брейна дает нам такие преимущества:
\begin{itemize}
  \item Проверка корректности построения выражений на уровне типов (невозможно написать выражение Lam 123 в пустом контексте, так как $\lambda$ захватывает только одну переменную).
  \item Можно абстрагировать это представление, превратив Scope в трансформер монад. Тогда нам остается лишь определить представителя класса Monad для нашего представления выражений (bind работает как подстановка), что делается крайне просто с точки зрения кодогенерации.
  \item Абстрактное представления дает нам обобщенные функции abstract и instantiate, которые абстрагируют переменную и инстанциируют самую внешнюю связную переменную соответственно. Таким образом решается проблема представления подстановок.
  \item Можно определить обобщенные Show и Eq --- не теряем простоты использования более простого представления без полиморфной рекурсии.
  \item С помощью механизма Deriving Haskell можно получить представителя классов Functor, Traversable и Foldable. Что дает нам функции toList --- список свободных переменных выражения --- и traverse --- применить аппликативную функцию к переменным выражения.

\end{itemize}
