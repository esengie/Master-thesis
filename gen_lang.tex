\section{Реализация}
В данной секции описана реализация языка спецификации языков с зависимыми типами.

\subsection{Парсер генераторы}
В ходе всей работы использовались лексер и парсер генераторы alex\cite{alex} и happy\cite{happy}.

Решение использовать именно парсер генераторы, а не парсер комбинаторы\cite{parsec} или другие методы парсинга было обуcловлено тем, что прогнозировались частые изменения грамматики вместе с эволюцией языка.

\begin{lstlisting}[caption={Часть спецификации парсера},captionpos=b]
Axiom   :   Header '=' '\t' Forall '\t'
            Premise '|---' JudgementNoEq '/t' '/t'
              { Axiom (snd $1) (fst $1) $4 $6 $8 }
        |   Header '=' '\t'
            Premise '|---' JudgementNoEq '/t'
              { Axiom (snd $1) (fst $1) [] $4 $6 }
\end{lstlisting}

Все изменения связанные с грамматикой языка проводились на уровне спецификации AST.

\subsection{Проверка корректного использования метапеременных}\label{toposort}
В секции~\ref{lang_spec} описывался язык и ограничения, налагаемые на спецификации.

Здесь описан алгоритм проверки использования метапеременных в контекстах других метапеременных. А если конкретнее --- проверки того, что метапеременные не используют метапеременных переданных правее в функциональном символе, который мы определяем.

Так как язык не обязывает пользователся явно передавать типы переменных метапеременных, используемых в функциональных символах, метапеременные могут быть не только аргументами определяемого функционального символа, но и типами термов предпосылок.

Вначале рассмотрим алгоритм в предположении того, что все метапеременные переданы нам в функциональный символ. Тогда единственные места, где должна проводится проверка --- это определения метапеременных. То есть предпосылки вида $x_1 : tm_1 \ldots x_k : tm_k  \vdash T$.

Давайте строить граф зависимости и проверять его на ацикличность. В предпосылке выше из $T$ будет исходить стрелки во все метапеременные $tm_i$.

Если же добавить предпослыки вида: $x_1 : tm_1 \ldots x_k : tm_k  \vdash t : T$, которые определяют $T$ и $t$, то мы ещё и добавляем стрелку из $t$ в $T$, так как $T$ используется в определении $t$.

Вообще говоря, все другие варианты линеаризуемы и можно проверять строгий порядок, а не частичный. Приведем последний возможный случай --- случай из-за которого введена топологическая сортировка --- $x_1 : tm_1 \ldots x_k : tm_k  \vdash tm : T$. Здесь мы ставим стрелки аналогично первому варианту, но сама метапеременная T не имеет фиксированной позиции в списке аргументов функционального символа из заключения.

Итак, мы построили граф зависимостей одних метапеременных от других. Для проверки корректности правила вывода мы делаем топологическую сортировку и проверяем, что наш граф является DAG'ом.

























%%%

\pagebreak
\subsection{Индексы де Брейна на уровне типов}\label{de_brujin_impl}
В нашем описании индексов де Брейна в Cекции~\ref{de_brujin} мы упомянули, что наивное их использование склонно к ошибкам и не использует систему типов Haskell.

Эту проблему можно решить с помощью полиморфной рекурсии\cite{Bird:Pat}. По сути, каждый раз когда мы абстрагируемся по переменной в представлении де Брейна, мы добавляем единицу ко всем связанным переменным внутри терма. Ключевым наблюдением является то, что мы можем добавлять единицу оборачивая терм в Maybe. Например:


\begin{lstlisting}[frame=single]
data Term a
  = Var a
  | App (Term a) (Term a)
  | Lam (Term (Maybe a))
\end{lstlisting}


Однако этот метод не очень удобен при кодогенерации, так как instance Monad будет зависет от определения Term. В той же статье предложен способ превращения этого паттерна программирования в трансформер монад\footnote{В коде Maybe заменен на Var, в соответствии со своей семантикой.}.


\begin{lstlisting}[frame=single]
data Var a = B | F a
newtype Scope f a = Scope { fromScope :: f (Var a) }

instance Monad f => Monad (Scope f) where
  return = Scope . return . F
  Scope m >>= f = Scope $ m >>= varAppWithDefault (return B) (fromScope . f)

instance MonadTrans Scope where
  lift = Scope . liftM F

\end{lstlisting}


Теперь мы можем написать общие функции абстрагирования по переменной и подстановки в самую внешнюю переменную терма.


\begin{lstlisting}[frame=single]
abstract :: (Functor f, Eq a) => a -> f a -> Scope f a
abstract x xs = Scope (fmap go xs) where
  go y = y <$ guard (x /= y)

instantiate :: Monad f => f a -> Scope f a -> f a
instantiate x (Scope xs) = xs >>= go where
  go B = x
  go (F y) = return y

\end{lstlisting}


При кодогенерации нам всего лишь понадобится определить гораздо более простую монаду подстановок для ADT термов, которые выглядят теперь так:


\begin{lstlisting}[frame=single]
data Term a
  = Var a
  | App (Term a) (Term a) (Scope Term a)
  | Lam (Term a) (Scope Term a)

instance Monad Term where
  Var v1 >>= f = f v1
  App v1 v2 >>= f = App (v1 >>= f) (v2 >>= f)
  Lam v1 v2 >>= f = Lam (v1 >>= f) (v2 >>>= f)

(>>>=) :: (Monad f) => Scope f a -> (a -> f b) -> Scope f b
m >>>= f = m >>= lift . f
\end{lstlisting}


Этот метод использован в библиотеке bound\cite{bound}. В виду того, что нам часто приходится заходить внутрь контекстов (это необходимо при приведении в нормальную форму), обобщенные индексы де Брейна используемые в bound нам не подходят. Это связано с тем, что мы не можем просто сопоставлять с образцом, нам нужно вызывать функцию fromScope, которая работает нетривиально. При реализации описанной выше fromScope соответствует паттерматчингу на терме.

\subsection{Построение термов}\label{build_exp}
Одной из проблем индексов де Брейна является их жесткая привязка к порядку переменных в контексте. Действительно чтобы переставить аргументы терма \lstinline{[Lam "y" (Lam "x" (App "x" (App "y" (App "y" "y"))))]} мы всего-лишь меняем их местами в моменты их связывания и получаем \lstinline{[Lam "x" (Lam "y" (App "x" (App "y" (App "y" "y"))))]}. Однако схожая операция для представления c использованием индексов де Брейна выливается в обход всего терма(!) \lstinline{[Lam (Lam (App 1 (App 2 (App 2 2))))]} превращается в \lstinline{[Lam (Lam (App 2 (App 1 (App 1 1))))]}.

Но если уж пользователь так написал спецификацию, что мы имеем терм с другим порядком переменных или терм с большим их количеством, то мы должны поменять эти переменные местами и даже попытаться удалить лишние переменные.

Например чтобы привести "(x y z).T" к "(z x).T". Мы должны удалить "y" и переставить "x" и "z" местами.

Так же мы поступаем при возможном расширении контекста нашей метапеременной, например имеем "S" и хотим построить "Lam A x.S" --- здесь нужна метапеременная "x.S", мы получаем её добавляя переменную в её контекст.

Решение предлагаемое в данной работе состоит из композиций операций \lstinline{swap_i'j}, \lstinline{remove_i} и \lstinline{add_i}. Каждая операция выполняет traverse терма, который мы меняем. Примеры функций:
\begin{lstlisting}[frame=single]
swap1'2 :: Var (Var  a) -> Identity (Var (Var  a))
swap1'2 (B ) = pure (F (B ))
swap1'2 (F (B )) = pure (B)
swap1'2 x = pure x

rem2 :: Var (Var a) -> TC (Var a)
rem2 B = pure B
rem2 (F B) = Left "There is var at 2"
rem2 (F (F x)) = pure (F x)

add2 :: Var a -> Identity (Var (Var a))
add2 B = pure $ B
add2 (F x) = pure $ F (F x)
\end{lstlisting}

Решение не является оптимальным, так как можно пройти по всему терму единожды и применить все эти операции сразу, но сложность генерации/написания такого кода возрастает значительно.

Для решения этой задачи написан модуль Solver\footnote{Стоит отметить что функции swap, rem и add должны быть сгенерированы и для этого ведется подсчёт в монаде кодогенерации путем записи максимального индекса. Следовательно функция swap дороже, так как мы генерируем $C_2^i$ функций. Именно поэтому алгоритм пытается использовать как можно меньше разных функций.}.

По сути мы либо имеем больший контекст и из него получаем меньший, либо наоборот. Хотим делать меньше swap'ов.

Рассмотрим случай приведения большего контекста к меньшему, "[x, y, z]" к "[y, x]". Мы идем справа налево, так как наиболее близкая связанная переменная наиболее правая. Удаляем те переменные которых нет в контексте к которому мы хотим прийти, таким образом обеспечиваем меньше вызовов к разным функциям rem\footnote{Мы не можем удалить переменную из контекста, если она присутствует в терме. Монада TC обеспечивает обработку ошибок удаления.}. Затем просто применяем алгоритм insertion sort на оставшихся контекстах. На количестве сгенерированных функций swap это не отразится.







%--

\subsection{Вывод типов и нормализация}
Сам infer работает как описано в Секции~\ref{typecheck}. Мы последовательно строим каждую предпосылку и вызываем функцию вывода типов или проверки типа выражения на равенство типу. Термы, которые мы передаем в эти функции, строим последовательно на основе переданных нам в функциональном символе или полученных при вызове функции вывода типов.

\begin{lstlisting}[caption={Пример правила вывода и части сгенерированной функции infer, соответствующей этому правилу},captionpos=b]
TApp =
    forall t1 : tm, t2 : tm, S : ty, x.T : ty
      |- t1 : pi(S, x.T),
      |- t2 : S,
      x : S |- T def
      |------------------------------
      |- app(t1 , t2, x.T) : T[x:=t2]

infer ctx (App v1 v2 v3)
  = do v4 <- infer ctx v2
       v5 <- pure (nf v4)
       v6 <- infer ctx v1
       checkEq (Pi v5 (toScope (fromScope v3))) v6
       checkT ctx TyDef v5
       checkT (consCtx v5 ctx) TyDef (fromScope v3)
       infer ctx v1
       infer ctx v2
       pure (instantiate v2 (toScope (fromScope v3)))
\end{lstlisting}

Стоит отметить, что порядок или количество переменных метапеременных которые у нас есть могут отличаться от порядка и вида контекста в котором наша метапеременная должна быть. Эту проблему мы решаем приводя метапеременные к контексту данному в секции forall правила вывода, методом описанным в Секции~\ref{build_exp}.

\begin{lstlisting}[caption={Искусственный пример случая несоответствия контекстов (контекст t нужно сократить до использования в предпосылке)},captionpos=b]
FRule =
    forall S : ty, t : tm
      |- t def
      |-----------------
      |- f(S, (x y).t) def
\end{lstlisting}

Функция nf пытается паттернматчиться на терме, если это не выходит, то данная редукция неприменима\footnote{Такой паттернматчинг невозможен с использованием библиотеки bound\cite{bound}, поэтому был написан модуль SimpleBound c обычными индексами де Брейна.}.


\subsection{Генерация кода}
Генерация кода происходит с использованием библиотеки haskell-src-exts\cite{src_exts}, которая дает нам функции генерации и манипуляции АСТ Haskell.

Тк большинство кода используемого для проверки не зависит от специфицированного языка, мы просто модифицируем написанный от руки модуль LangTemplate. В нем нужно определить функции приведения в нормальную форму и вывода типов. Также нужно определить тип данных термов и определить монадическое действие на типе данных термов.

Всё остальное либо генерируется с помощью Template Haskell\cite{TH} --- instance Traversable, Functor, Eq, Show, Foldable\footnote{Foldable дает нам функцию toList, которая возвращает свободные переменные терма, Traversable позволяет применять функции swap, rem и add  к переменным обходя весь терм.}, либо написано от руки с вызовами функций nf или infer.

\begin{lstlisting}[caption={Проверка типов и контексты},captionpos=b]
emptyCtx :: (Show a, Eq a) => Ctx a
emptyCtx x = Left $ "Variable not in scope: " ++ show x

consCtx :: (Show a, Eq a) => Type a -> Ctx a -> Ctx (Var a)
consCtx ty ctx B = pure (F <$> ty)
consCtx ty ctx (F a)  = (F <$>) <$> ctx a

checkT :: (Show a, Eq a) => Ctx a -> Type a -> Term a -> TC ()
checkT ctx want t = do
  have <- infer ctx t
  when (nf have /= nf want) $ Left $
    "type mismatch, have: " ++ (show have) ++ " want: " ++ (show want)
\end{lstlisting}







%--
