\section{Реализация}
В данной секции описана реализация языка спецификации языков с зависимыми типами.

\subsection{Парсер генераторы}
В ходе всей работы использовались лексер и парсер генераторы alex\cite{alex} и happy\cite{happy}.

Решение использовать именно парсер генераторы, а не парсер комбинаторы\cite{parsec} или другие методы парсинга было обуcловлено тем, что прогнозировались частые изменения грамматики вместе с эволюцией языка.

\begin{lstlisting}[caption={Часть спецификации парсера},captionpos=b]
Axiom   :   Header '=' '\t' Forall '\t'
            Premise '|---' JudgementNoEq '/t' '/t'
              { Axiom (snd $1) (fst $1) $4 $6 $8 }
        |   Header '=' '\t'
            Premise '|---' JudgementNoEq '/t'
              { Axiom (snd $1) (fst $1) [] $4 $6 }
\end{lstlisting}

\subsection{Индексы де Брейна и их проблемы(задачки с индексами)}\label{de_brujin_impl}

\subsection{Построение термов}\label{build_exp}

Внутри кодогенерации функций nf и infer мы должны уметь строить термы языка, который мы проверяем/редуцируем. В функции infer это связано с тем, что при проверке предпосылок и возврате типа конструкции мы должны уметь строить произвольные термы специфицированного языка. В функции nf это связано с тем, что правая часть редукции может содержать произвольные термы языка.

Итак, на данной стадии работы алгоритма у нас имеется ассоциативный массив сопоставляющий метаперенные с выражениями на Haskell, которые им соответствуют (это всегда просто переменные языка). Рассмотрим дальнейший ход действий на примере. Предположим дана аксиома ff (см. вставку~\ref{ffrulee}).

\begin{lstlisting}[label={ffrulee}, caption={Искусственное правило вывода для конструкции ff},captionpos=b, frame=single, float, floatplacement=H]
FRule =
    forall S : ty, t : tm, T : ty
      x:S, y:S |- t : T,
      x:T |- t : bool,
      |- gf(S, (x z).rf(T, (y r).T)) : rf(S, (x z).T)
      |-----------------
      |- ff(S, t) def
\end{lstlisting}

На момент вызова у нас есть S и t в пустом контексте. Мы уже отсортировали предпосылки по трем группам описанным в~\ref{repr}. Сперва нам нужны предпослыки вводящие метапеременные. Поэтому первой предпосылкой которую мы проверяем является \lstinline{x:S, y:S |- t : T}. Чтобы поучить терм T мы должны вызвать функцию вывода типов в контексте, который нам передан, расширенном двумя вхождениями типа S. Ещё мы должны проверить, что эти расширения определены, то есть вызывать функции infer в постепенно увеличивающихся контекстах с термом, который мы хотим добавить в контекст в качестве аргумента.

Для этого мы должны расширить контекст терма t. Это является ограничением, наложенным на нас нашим же представлением, так как иначе у нас не сойдутся типы. Также могло случится так, что мы должны были бы переставить наши переменные в контексте.

Затем, получив нашу переменную в увеличенном контексте (в forall она имеет контекст меньшей длины), до того как мы добавим переменную, которая является её представителем в коде Haskell, мы должны уменьшить её контекст до того, что указан в forall\footnote{Не обязателен тот же порядок контекста, тк мы все равно о нём заботимся во время построения термов, то есть мы можем хранить в массиве представление метапеременной (z y x).T вместо (x y z).T. Главное чтобы это было указано в структуре, которую мы храним.}.

Затем мы проверяем вторую предпослыку, аналогично описанному выше способу. В третей предпосылке мы должны строить терм \lstinline{gf(S, (x z).rf(T, (y r).T))}. Это делается рекурсивно внутри монады кодогенерации, чтобы мы имели доступ к нашему ассоциативному массиву метапеременных. В данном примере нам потребуется из x.T получить (x z).T и (x z y r).T. Затем все аналогично.

Но при получении типа мы должны проверить его на равенство типу  \lstinline{rf(S, (x z).T)}. Который мы строим аналогично предыдущему описанию, затем вызываем функцию проверки равенства типов, на терме полученном при выводе типа \lstinline{gf(S, (x z).rf(T, (y r).T))} и построенном из метапеременных терма \lstinline{rf(S, (x z).T)} (см. вставку~\ref{FRule}).

\hfill

Одной из проблем индексов де Брейна является их жесткая привязка к порядку переменных в контексте. Действительно чтобы переставить аргументы терма \lstinline{[Lam ``y'' (Lam ``x'' (App ``x'' (App ``y'' (App ``y'' ``y''))))]} мы всего лишь меняем их местами в моменты их связывания и получаем \lstinline{[Lam ``x'' (Lam ``y'' (App ``x'' (App ``y'' (App ``y'' ``y''))))]}. Однако схожая операция для представления c использованием индексов де Брейна выливается в обход всего терма(!) \lstinline{[Lam (Lam (App 1 (App 2 (App 2 2))))]} превращается в \lstinline{[Lam (Lam (App 2 (App 1 (App 1 1))))]}.

Но если уж пользователь так написал спецификацию, что мы имеем терм с другим порядком переменных или терм с большим их количеством, то мы должны поменять эти переменные местами и даже попытаться удалить лишние переменные.

Например, чтобы привести ``(x y z).T'' к ``(z x).T''. Мы должны удалить ``y'' и переставить ``x'' и ``z'' местами.

Так же мы поступаем при возможном расширении контекста нашей метапеременной, например имеем ``S'' и хотим построить ``Lam A x.S'' --- здесь нужна метапеременная ``x.S'', мы получаем её добавляя переменную в её контекст.

Решение предлагаемое в данной работе состоит из композиций операций \lstinline{swap_i'j}, \lstinline{remove_i} и \lstinline{add_i}. Каждая операция выполняет traverse терма, который мы меняем. Примеры во вставке~\ref{swrem}.

\begin{lstlisting}[label={swrem}, caption={Примеры функций},captionpos=b, frame=single, float, floatplacement=H]
swap1'2 :: Var (Var  a) -> Identity (Var (Var  a))
swap1'2 (B ) = pure (F (B ))
swap1'2 (F (B )) = pure (B)
swap1'2 x = pure x

rem2 :: Var (Var a) -> TC (Var a)
rem2 B = pure B
rem2 (F B) = Left "There is var at 2"
rem2 (F (F x)) = pure (F x)

add2 :: Var a -> Identity (Var (Var a))
add2 B = pure $ B
add2 (F x) = pure $ F (F x)
\end{lstlisting}

Решение не является оптимальным, так как можно пройти по всему терму единожды и применить все эти операции сразу, но сложность генерации/написания такого кода возрастает значительно.

Для решения этой задачи написан модуль Solver\footnote{Стоит отметить что функции swap, rem и add должны быть сгенерированы и для этого ведется подсчёт в монаде кодогенерации путем записи максимального индекса. Следовательно функция swap дороже, так как мы генерируем $C_2^i$ функций. Именно поэтому алгоритм пытается использовать как можно меньше разных функций.}.

По сути мы либо имеем больший контекст и из него получаем меньший, либо наоборот. Хотим делать меньше swap'ов.

Рассмотрим случай приведения большего контекста к меньшему, ``[x, y, z]'' к ``[y, x]''. Мы идем справа налево, так как наиболее близкая связанная переменная наиболее правая. Удаляем те переменные которых нет в контексте к которому мы хотим прийти, таким образом обеспечиваем меньше вызовов к разным функциям rem\footnote{Мы не можем удалить переменную из контекста, если она присутствует в терме. Монада TC обеспечивает обработку ошибок удаления.}. Затем просто применяем алгоритм insertion sort на оставшихся контекстах. На количестве сгенерированных функций swap это не отразится.







%--

\subsection{Вывод типов и нормализация}\label{nf_infer}
Сам infer работает как описано в Разделе~\ref{typecheck}. Мы последовательно строим каждую предпосылку и вызываем функцию вывода типов или проверки типа выражения на равенство типу. Термы, которые мы передаем в эти функции, строим последовательно на основе переданных нам в конструкции или полученных при вызове функции вывода типов (подробно описано в Разделе~\ref{build_exp}).

\begin{lstlisting}[caption={Пример правила вывода и части сгенерированной функции infer, соответствующей этому правилу},captionpos=b, frame=single, float, floatplacement=H]
If-then =
    forall t : tm, t1 : tm, t2 : tm, x.A : ty
      x : bool |- A def,   -- (1)
      |- t1 : A[x:=true],  -- (2)
      |- t2 : A[x:=false], -- (3)
      |- t : bool          -- (4)
      |-------------------------------
      |- if(x.A, t, t1, t2) : A[x:=t]

infer ctx (If v1 v2 v3 v4) = do
     -- check (4) ------------------------------------------
     v5 <- infer ctx v2
     checkEq Bool v5
     -- check (3) ------------------------------------------
     v6 <- infer ctx v4
     checkEq (instantiate False (toScope (fromScope v1))) v6
     -- check (2) ------------------------------------------
     v7 <- infer ctx v3
     checkEq (instantiate True (toScope (fromScope v1))) v7
     -- check (1) ------------------------------------------
     checkT ctx TyDef Bool
     checkT (consCtx Bool ctx) TyDef (fromScope v1)
     -- return ---------------------------------------------
     pure (instantiate v2 (toScope (fromScope v1)))
\end{lstlisting}

Стоит отметить, что порядок или количество переменных метапеременных, которые нам переданы, могут отличаться от порядка и вида контекста, в котором наша метапеременная должна быть в момент её использования. Эту проблему мы решаем приводя метапеременные к контексту данному в секции forall нашего правила вывода методом, описанным в Секции~\ref{build_exp}.

\begin{lstlisting}[caption={Искусственный пример случая несоответствия контекстов: контекст t нужно сократить до использования в предпосылке.},label={FRule},captionpos=b, frame=single, float, floatplacement=H]
FRule =
    forall S : ty, t : tm, T : ty
      x:S, y:S |- t : T, -- (1)
      x:T |- t : bool,   -- (2)
      |- gf(S, (x z).rf(T, (y r).T)) : rf(S, (x z).T) -- (3)
      |-----------------
      |- ff(S, t) def

 infer :: (Show a, Eq a) => Ctx a -> Term a -> TC (Type a)
 infer ctx (Ff v1 v2) = do
     -- check (1) -----------------------------------------
     checkT ctx TyDef v1
     checkT (consCtx v1 ctx) TyDef (rt add1 v1)
     v3 <- infer (consCtx (rt add1 v1) (consCtx v1 ctx))
             (rt add1 (rt add1 v2))
     -- check (3) -----------------------------------------
     v4 <- pure (nf v3) >>= traverse rem1 >>= traverse rem1
     v5 <- infer ctx
             (Gf v1
                (toScope2
                   (Rf (rt add1 (rt add1 v4))
                      (toScope2 (rt add1 (rt add1 (rt add1 (rt add1 v4))))))))
     checkEq (Rf v1 (toScope2 (rt add1 (rt add1 v4)))) v5
     -- check (2) -----------------------------------------
     checkT ctx TyDef v4
     v6 <- infer (consCtx v4 ctx) (rt add1 v2)
     checkEq Bool v6
     -- return --------------------------------------------
     pure TyDef
\end{lstlisting}

Функция nf пытается сопоставиться с образцом на терме, если это не выходит, то данная редукция неприменима. Поэтому нужен способ отслеживать какие редукции уже были опробованы, а какие нет. Это делается с помощью специальной структуры данных Cnt, описанной во вставке~\ref{lst_nf}. Эта структура служит в качестве целого числа, которое мы инициализируем количеством редукций применимых к нашей кострукции языка, и каждая несовпавшая редукция вызывает функцию nf c меньшим и меньшим числом, если же мы доходим до нуля, то мы истощили набор применимых редукций --- а значит терм находится в нормальной форме.

Стоит отметить, что такое сопоставление с образцом невозможно с использованием библиотеки bound\cite{bound}, поэтому был написан модуль SimpleBound c обычными, а не обобщенными, индексами де Брейна.

\begin{lstlisting}[caption={Приведение в нормальную форму пытается применить все редукции данного функционального символа},captionpos=b, frame=single, float, floatplacement=H, label={lst_nf}]
data Cnt = Bot | U (Cnt)
  deriving(Eq, Show)

nf :: (Show a, Eq a) => Term a -> Term a
nf (If v1 v2 v3 v4)
  = nf' (U (U Bot)) (If (nf1 v1) (nf v2) (nf v3) (nf v4))

nf':: (Show a, Eq a) => Cnt -> Term a -> Term a
nf' (U (U _)) al@(If (Scope v1) True v2 v3)
  = case
      do v4 <- pure v1
         v5 <- pure v2
         v6 <- pure v3
         pure v5
      of
        Left _ -> nf' (U Bot) al
        Right x -> nf x
nf' (U _) al@(If (Scope v1) False v2 v3)
  = case
      do v4 <- pure v1
         v5 <- pure v2
         v6 <- pure v3
         pure v6
      of
        Left _ -> nf' Bot al
        Right x -> nf x
nf' _ x = x
\end{lstlisting}


\subsection{Генерация кода}
Генерация кода происходит с использованием библиотеки haskell-src-exts\cite{src_exts}, которая дает нам функции генерации и манипуляции АСТ Haskell.

Тк большинство кода используемого для проверки не зависит от специфицированного языка, мы просто модифицируем написанный от руки модуль LangTemplate. В нем нужно определить функции приведения в нормальную форму и вывода типов. Также нужно определить тип данных термов и определить монадическое действие на типе данных термов.

Всё остальное либо генерируется с помощью Template Haskell\cite{TH} --- instance Traversable, Functor, Eq, Show, Foldable\footnote{Foldable дает нам функцию toList, которая возвращает свободные переменные терма, Traversable позволяет применять функции swap, rem и add  к переменным обходя весь терм.}, либо написано от руки с вызовами функций nf или infer.

\begin{lstlisting}[caption={Проверка типов и контексты},captionpos=b]
emptyCtx :: (Show a, Eq a) => Ctx a
emptyCtx x = Left $ "Variable not in scope: " ++ show x

consCtx :: (Show a, Eq a) => Type a -> Ctx a -> Ctx (Var a)
consCtx ty ctx B = pure (F <$> ty)
consCtx ty ctx (F a)  = (F <$>) <$> ctx a

checkT :: (Show a, Eq a) => Ctx a -> Type a -> Term a -> TC ()
checkT ctx want t = do
  have <- infer ctx t
  when (nf have /= nf want) $ Left $
    "type mismatch, have: " ++ (show have) ++ " want: " ++ (show want)
\end{lstlisting}







%--
