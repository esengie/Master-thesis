\documentclass{beamer}
\mode<presentation> {
% The Beamer class comes with a number of default slide themes
% which change the colors and layouts of slides. Below this is a list
% of all the themes, uncomment each in turn to see what they look like.
%\usetheme{default}
%\usetheme{AnnArbor}
%\usetheme{Antibes}
%\usetheme{Bergen}
%\usetheme{Berkeley}
%\usetheme{Berlin}
%\usetheme{Boadilla}
%\usetheme{CambridgeUS}
%\usetheme{Copenhagen}
%\usetheme{Darmstadt}
%\usetheme{Dresden}
%\usetheme{Frankfurt}
%\usetheme{Goettingen}
%\usetheme{Hannover}
%\usetheme{Ilmenau}
%\usetheme{JuanLesPins}
%\usetheme{Luebeck}
\usetheme{Madrid}
% \usetheme{Malmoe}
%\usetheme{Marburg}
%\usetheme{Montpellier}
%\usetheme{PaloAlto}
%\usetheme{Pittsburgh}
%\usetheme{Rochester}
%\usetheme{Singapore}
%\usetheme{Szeged}
%\usetheme{Warsaw}
\setbeamertemplate{itemize items}[square]
\setbeamertemplate{enumerate items}[square]
% As well as themes, the Beamer class has a number of color themes
% for any slide theme. Uncomment each of these in turn to see how it
% changes the colors of your current slide theme.

%\usecolortheme{albatross}
%\usecolortheme{beaver}
%\usecolortheme{beetle}
%\usecolortheme{crane}
%\usecolortheme{dolphin}
%\usecolortheme{dove}
%\usecolortheme{fly}
%\usecolortheme{lily}
%\usecolortheme{orchid}
%\usecolortheme{rose}
%\usecolortheme{seagull}
%\usecolortheme{seahorse}
%\usecolortheme{whale}
%\usecolortheme{wolverine}

%\setbeamertemplate{footline} % To remove the footer line in all slides uncomment this line
%\setbeamertemplate{footline}[page number] % To replace the footer line in all slides with a simple slide count uncomment this line
\setbeamertemplate{navigation symbols}{} % To remove the navigation symbols from the bottom of all slides uncomment this line
}

\usepackage{inputenc}
\usepackage{listings}
\usepackage{polyglossia}
\setdefaultlanguage{russian}
\usepackage{graphicx} % Allows including images
\usepackage{booktabs} % Allows the use of \toprule, \midrule and \bottomrule in tables

\usepackage{amssymb,amsfonts,amsmath,mathtext}
\usepackage{bussproofs}
\usepackage{stmaryrd}

\newenvironment{scprooftree}[1]%
  {\gdef\scalefactor{#1}\begin{center}\proofSkipAmount \leavevmode}%
  {\scalebox{\scalefactor}{\DisplayProof}\proofSkipAmount \end{center} }


\usepackage{fontspec}
\setmainfont[Mapping=tex-text]{CMU Serif}
\setsansfont{CMU Sans Serif}
\setmonofont{CMU Typewriter Text}
\usepackage{mdwlist}
% \usepackage{enumitem}

% \newcommand\tab[1][0.8cm]{\hspace*{#1}}

\newcommand{\IFF}{\begin{scprooftree}{0.92}
\AxiomC{$\Gamma, x : Bool \vdash T$}
\AxiomC{$\Gamma \vdash t : Bool$}
\AxiomC{$\Gamma \vdash a : T[x:=True]$}
\AxiomC{$\Gamma \vdash b : T[x:=False]$}
\QuaternaryInfC{$\Gamma \vdash if(t, T, a, b) : T[x:=t] $}
% \DisplayProof
\end{scprooftree}
}

%----------------------------------------------------------------------------------------
%	TITLE PAGE
%----------------------------------------------------------------------------------------

\title[Ген. зависимых языков]{Генерация зависимых языков по спецификации пользователя} % The short title appears at the bottom of every slide, the full title is only on the title page
\author[Гарифуллин Шамиль]
{Гарифуллин Шамиль Раифович\\
{\small Научный руководитель: Исаев Валерий Иванович}
}
\institute[СПбАУ]{СПбАУ}
\date{\today} % Date, can be changed to a custom date

\begin{document}

\begin{frame}
\titlepage % Print the title page as the first slide
\end{frame}

% \begin{frame}
% \frametitle{Overview} % Table of contents slide, comment this block out to remove it
% \tableofcontents % Throughout your presentation, if you choose to use \section{} and \subsection{} commands, these will automatically be printed on this slide as an overview of your presentation
% \end{frame}
%------------------------------------------------
% \section{First Section} % Sections can be created in order to organize your presentation into discrete blocks, all sections and subsections are automatically printed in the table of contents as an overview of the talk
%------------------------------------------------
% \subsection{Subsection Example} % A subsection can be created just before a set of slides with a common theme to further break down your presentation into chunks

%----------------------------------------------------------------------------------------
%	PRESENTATION SLIDES
%----------------------------------------------------------------------------------------
%----------------------------------------------------------------------------------------
%	PRESENTATION SLIDES
%----------------------------------------------------------------------------------------

\begin{frame}[fragile]
\frametitle{Введение}

Языки с зависимыми типами --- типы могут зависеть от термов.

Одна из частых ошибок при программировнии на языке Haskell --- взятие первого элемента пустого списка.

\begin{verbatim}
head :: [a] -> a
head (x:_) = x
head [] = error "No head!"
\end{verbatim}

В языке с зависимыми типами мы можем усилить ограничения на входные данные функции.

\begin{verbatim}
head :: {n : N} -> Vec a (suc n) -> a
head (x:_) = x
\end{verbatim}

\end{frame}

%------------------------------------------------
\begin{frame}
\frametitle{Применение языков с зависимыми типами}

\begin{itemize}
\item Одно из основных применений языков с зависимыми это доказательство утверждений
\item Конструкции нужные нам могут отстутствовать в языке
\item Решение --- реализовать свой язык с нужными конструкциями
\item Для проверки равенств в типах нужно уметь нормализовывать термы языка.
\textit{Нормализация} --- применение редукций для переписывания терма, пока это возможно
\item $(\lambda x \rightarrow fib(x)) 3 == 1 + 5$?
\item Алгоритм всегда один и тот же --- есть возможность кодогенерации
\end{itemize}
\end{frame}
%------------------------------------------------

\begin{frame}
\frametitle{Цели и задачи}
Реализовать генерацию алгоритма проверки типов и вычислителя зависимых языков по спецификации

\begin{itemize}
\item Разработка языка спецификации и налагемых им ограничений
\item Выбор внутреннего представления АСД и генерация структур данных конструкций языка
\item Генерация кода функций проверки типов и нормализации
\end{itemize}
\end{frame}
%------------------------------------------------

\begin{frame}
\frametitle{Спецификация языка}
Язык программирования состоят из
\begin{itemize}
\item Конструкций
\item Правил построения конструкций (правил вывода)
\item Правил вычисления (редукций)
\end{itemize}
\end{frame}
%------------------------------------------------
\begin{frame}[fragile]
\begin{columns}[T] % align columns
\begin{column}{.48\textwidth}

\begin{center}
\AxiomC{}
\UnaryInfC{$\vdash$}
\DisplayProof
\quad
\AxiomC{$\Gamma \vdash A$}
\RightLabel{, $x \notin \Gamma$}
\UnaryInfC{$\Gamma, x : A \vdash$}
\DisplayProof
\quad
\AxiomC{$\Gamma \vdash$}
\RightLabel{, $x : A \in \Gamma$}
\UnaryInfC{$\Gamma \vdash x : A$}
\DisplayProof
\end{center}

\medskip
\begin{center}
\AxiomC{$\Gamma \vdash a : A$}
\AxiomC{$\Gamma \vdash B$}
\RightLabel{, $A \equiv B$}
\BinaryInfC{$\Gamma \vdash a : B$}
\DisplayProof
\end{center}

\end{column}%

\hfill%

\begin{column}{.48\textwidth}

\begin{verbatim}
Наше представление
\end{verbatim}

\end{column}%
\end{columns}

\end{frame}
%------------------------------------------------
\begin{frame}
\frametitle{Ограничения накладываемые языком?}
\begin{itemize}
\item Топосорт
\item Проверка типов и проч
\item Идея в том, что этим мы отличаемся от всяких LF
\end{itemize}
\end{frame}
%------------------------------------------------
\begin{frame}
\frametitle{Внутреннее представление}
Индексы де Брейна?
\end{frame}
%------------------------------------------------
\begin{frame}
\frametitle{Генерация кода}
Индексы де Брейна?
\end{frame}
%------------------------------------------------
\begin{frame}[fragile]
\frametitle{Пример кода?}

\end{frame}

\begin{frame}
\frametitle{Результаты}
Реализована генерация алгоритма проверки типов и вычислителя зависимых языков по спецификации.

\begin{itemize}
\item Спроектирован типизированный язык спецификации
\item Генерация структур данных конструкций языка с использованием индексов де Брейна на уровне типов с использованием полиморфной рекурсии
\item Генерация кода функции проверки типов термов специфицированного языка с передачей контекстов свободных переменных и функции нормализации на основе сопоставления с образцом
\end{itemize}

Репозиторий проекта: \url{github.com/esengie/fpl-exploration-tool/}
\end{frame}

%------------------------------------------------
\end{document}
