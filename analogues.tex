\section{Обзор аналогов}
Построение языков программирования с зависимыми типами по спецификации является задачей достаточно специфичной. Ниже перечислены инструменты, применяемые в похожих ситуациях (изучение формальных систем, языков программирования и их реализация).

\subsection{BNFC}
Чем-то похожим средством разработки является BNFC\cite{bnfc}. Позволяет генерировать фронтенд компилятора по аннотированной грамматике языка в форме Бэкуса-Наура\cite{lbnf}.

Генерирует лексер, парсер и преттипринтер языка. Также создает АСТ и заготовку для написания редукций (просто один большой case).

\subsection{PLT/Redex}
PLT/Redex\cite{plt:redex} --- EDSL Racket созданный для спецификации и изучения операционных семантик. Может быть использован для спецификации языков программирования, в том числе и с зависимыми типами.

Позволяет рандомизированно тестировать цикличность редукций или иные свойства языка. Также позволяет визуализировать редукции.

Однако спецификация языков с зависимыми типами занимает столько же усилий, как если бы пользователь писал реализацию языка в Haskell\cite{plt:ex}. Большую сложность составляют подстановки --- проблема, обойденная в данной работе с использованием представления в виде индексов де Брейна.

\subsection{Twelf}
Twelf\cite{twelf} является реализацией LF\cite{Pfenning2002}. Используется для спецификации и доказательств свойств логик и языков программирования.

Для спецификации задаются высказывания языка (используется принцип "высказывания в качестве типов"\cite{harper:1993}) и некоторые операции над ним в виде отнощений. Затем доказываются  $\forall\Sigma$-свойства данного языка.

Таким образом, в Twelf можно доказывать свойства спецификаций языков или приводить спецификации к форме, в которой выполняются интересующие нас свойства. Затем можно использовать программу описанную в данной работе для имплементации языка.






%--
