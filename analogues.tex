\section{Обзор аналогов}
Построение языков программирования с зависимыми типами по спецификации является задачей достаточно специфичной. Ниже перечислены некотороые инструменты, применяемые в похожих ситуациях. Таких, как изучение формальных систем, языков программирования и их реализация.

\subsection{BNFC}
Похожим на программу описанную в дипломной работе средством разработки является BNFC\cite{bnfc}. Эта утилита позволяет генерировать фронтенд компилятора по аннотированной грамматике языка в форме Бэкуса-Наура\cite{lbnf}. \textit{Фронтендом} называется комбинация представления АСД и синтаксического и лексического анализаторов языка.

Программа генерирует лексический анализатор, синтаксический анализатор и вывод структур на экран языка заданного в спецификации. Также она генерирует абстрактное синтаксическое дерево и заготовку для написания редукций, представленную в виде большой конструкции switch языка С или её аналогов.

Генерирует представления на C, C++, C\#, Haskell, Java и OCaml.

\subsection{PLT/Redex}
PLT/Redex\cite{plt:redex} --- встроенный DSL на языке Racket, созданный для спецификации и изучения операционных семантик языков программирования. Он используется для спецификации языков программирования, в том числе и с зависимыми типами.

Из отличительных черт: позволяет случайным образом тестировать цикличность редукций или иные свойства языка, задаваемые пользователем в DSL. Также позволяет визуализировать порядок редукций.

Однако описание языков с зависимыми типами не является лишь спецификацией, а требует ещё и реализации пользователя\cite{plt:ex}. Некоторую сложность составляют подстановки --- проблема, обойденная в данной дипломной работе благодаря использованию представления выражений языка в виде индексов де Брейна.

\subsection{Twelf}
Twelf\cite{twelf} является реализацией LF\cite{Pfenning2002}. Эта программа используется для спецификации и доказательств свойств логик и языков программирования.

В спецификации задаются высказывания языка (используется принцип ``высказывания в качестве типов''\cite{harper:1993}) и некоторые операции над языком в виде отношений на языке Twelf. Затем доказываются свойства вида  $\forall\Sigma$ специфицированного языка.

Таким образом, в Twelf можно доказывать свойства спецификаций языков или приводить спецификации к форме, в которой выполняются свойства интересующие нас. Затем можно использовать программу, описанную в данной дипломной работе, для реализации языка.






%--
