%----------------------------------------------------------------------------------------
%	PRESENTATION SLIDES
%----------------------------------------------------------------------------------------

\begin{frame}[fragile]
\frametitle{Введение}
Языки с зависимыми типами --- типы могут зависеть от термов.

Одна из частых ошибок при программировнии на языке Haskell --- взятие первого элемента пустого списка.

\begin{verbatim}
head :: [a] -> a
head (x:_) = x
head [] = error "No head!"
\end{verbatim}

Эту проблему легко решить

\begin{verbatim}
head :: {n : N} -> Vec a (suc n) -> a
head (x:_) = x
\end{verbatim}


\end{frame}

%------------------------------------------------

\begin{frame}
\frametitle{Цели и задачи}
Реализовать генерацию тайпчекера и вычислителя языка по спецификации.

\begin{itemize}
\item Дизайн языка спецификации
\item Выбор внутреннего представления и генерация структур данных конструкций языка
\item Генерация тайпчекера и функции нормализации
\end{itemize}
\end{frame}
%------------------------------------------------
\begin{frame}
\frametitle{Аналоги?}
Пример норм языка
\end{frame}
%------------------------------------------------
\begin{frame}
\frametitle{Язык спецификации}
\begin{center}
\AxiomC{$\Gamma, x : S \vdash T\ type $}
\AxiomC{$\Gamma, \vdash f : pi(S, T) $}
\AxiomC{$\Gamma \vdash t : S $}
\TrinaryInfC{$\Gamma \vdash app(f, t, T) : T[x:=t]$}
\DisplayProof
\end{center}
Пример норм языка
\end{frame}
%------------------------------------------------
\begin{frame}[fragile]
\frametitle{Язык спецификации}
\begin{verbatim}
Наше представление
\end{verbatim}
\end{frame}
%------------------------------------------------
\begin{frame}
\frametitle{Ограничения накладываемые языком?}
\begin{itemize}
\item Топосорт
\item Проверка типов и проч
\item Идея в том что этим мы отличаемся от всяких LF
\end{itemize}
\end{frame}
%------------------------------------------------
\begin{frame}
\frametitle{Внутреннее представление}
Индексы де Брейна?
\end{frame}
%------------------------------------------------
\begin{frame}
\frametitle{Генерация кода}
Индексы де Брейна?
\end{frame}
%------------------------------------------------
\begin{frame}[fragile]
\frametitle{Пример кода?}

\end{frame}
