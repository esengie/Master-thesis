\appendix
\section*{Приложения}
\addcontentsline{toc}{section}{Приложения}
\renewcommand{\thesubsection}{\Alph{subsection}}

\subsection{Доказательство корректности функции sort}\label{sort_proof}

Ниже показан пример доказательства того, что функция filter выдает подсписок исходного списка.
Код написан на Agda\cite{agda}

\begin{lstlisting}
-- Определяем предикат принадлежности элемента списку.
data _∈_ {A : Set} (a : A) : List A → Set where
  here : (xs : List A) -> a ∈ (a ∷ xs)
  there : (x : A) (xs : List A) -> a ∈ xs -> a ∈ (x ∷ xs)

-- Определяем предикат xs ⊆ ys, означающий "список xs является подсписком ys".
data _⊆_ {A : Set} : List A -> List A → Set where
    nil : [] ⊆ []
    larger : {y : A} {xs ys : List A} -> xs ⊆ ys -> xs ⊆ (y ∷ ys)
    cons  : {x : A} {xs ys : List A} -> xs ⊆ ys -> (x ∷ xs) ⊆ (x ∷ ys)

-- Докажем, что filter xs ⊆ xs для любого списка xs.
filter' : {A : Set} -> (A → Bool) → List A → List A
filter' p [] = []
filter' p (x ∷ xs) = if p x then x ∷ filter' p xs else filter' p xs

filterLess : {A : Set} -> (p : A -> Bool) -> (xs : List A) -> filter' p xs ⊆ xs
filterLess p [] = nil
filterLess p (x ∷ xs) with p x
filterLess p (x ∷ xs) | false = larger (filterLess p xs)
filterLess p (x ∷ xs) | true = cons (filterLess p xs)

\end{lstlisting}


%--
