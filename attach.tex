\appendix
\section*{Приложения}
% \addcontentsline{toc}{section}{Приложения}
\renewcommand{\thesubsection}{\Alph{subsection}}

\subsection{Доказательство корректности функции filter}\label{sort_proof}

Ниже показан пример доказательства того, что функция filter выдает подсписок исходного списка.
Код написан на Agda\cite{agda}.

\begin{lstlisting}[caption={Определяем предикат означающий "список xs является подсписком ys"},captionpos=b, frame=single]
data _in_ {A : Set} : List A → List A → Set where
    nil : [] in []
    larger : {y : A} {xs ys : List A} → xs in ys → xs in (y :: ys)
    cons  : {x : A} {xs ys : List A} → xs in ys → (x :: xs) in (x :: ys)
\end{lstlisting}


\begin{lstlisting}[caption={Докажем, что filter xs подсписок xs для любого списка xs},captionpos=b, frame=single]
filter' : {A : Set} → (A → Bool) → List A → List A
filter' p [] = []
filter' p (x :: xs) = if p x then x :: filter' p xs else filter' p xs

filterLess : {A : Set} → (p : A → Bool) → (xs : List A) → filter' p xs in xs
filterLess p [] = nil
filterLess p (x :: xs) with p x
filterLess p (x :: xs) | false = larger (filterLess p xs)
filterLess p (x :: xs) | true = cons (filterLess p xs)
\end{lstlisting}


\subsection{Ссылка на исходный код}\label{source_code}
Имплементация работы описанной в дипломе находится на \href{https://github.com/esengie/fpl-exploration-tool}{репозитории на гитхаб}: \url{github.com/esengie/fpl-exploration-tool}

\subsection{Спецификация $\lambda\Pi$ с булевыми выражениями и сгенерирванный код}\label{lambdaPiSpec}

% redefine \VerbatimInput
\RecustomVerbatimCommand{\VerbatimInput}{VerbatimInput}%
{
 %
 frame=lines,  % top and bottom rule only
 framesep=2em, % separation between frame and text
 rulecolor=\color{Gray},
 %
 labelposition=topline,
 %
 commentchar=*        % comment character
}

\VerbatimInput[label=\fbox{\color{Black}lambdaPi.fpl}]{"code/lambdaPi.fpl"}
\VerbatimInput[label=\fbox{\color{Black}Lang.hs}]{"code/Lang.hs"}

%--
