\appendix
\section*{Приложения}
% \addcontentsline{toc}{section}{Приложения}
\renewcommand{\thesubsection}{\Alph{subsection}}

\subsection{Доказательство корректности функции filter}\label{sort_proof}

Ниже показан пример доказательства того, что функция filter выдает подсписок исходного списка.
Код написан на Agda\cite{agda}.

\begin{lstlisting}[caption={Определяем предикат означающий "список xs является подсписком ys"},captionpos=b]
data _in_ {A : Set} : List A → List A → Set where
    nil : [] in []
    larger : {y : A} {xs ys : List A} → xs in ys → xs in (y :: ys)
    cons  : {x : A} {xs ys : List A} → xs in ys → (x :: xs) in (x :: ys)
\end{lstlisting}


\begin{lstlisting}[caption={Докажем, что filter xs подсписок xs для любого списка xs},captionpos=b]
filter' : {A : Set} → (A → Bool) → List A → List A
filter' p [] = []
filter' p (x :: xs) = if p x then x :: filter' p xs else filter' p xs

filterLess : {A : Set} → (p : A → Bool) → (xs : List A) → filter' p xs in xs
filterLess p [] = nil
filterLess p (x :: xs) with p x
filterLess p (x :: xs) | false = larger (filterLess p xs)
filterLess p (x :: xs) | true = cons (filterLess p xs)
\end{lstlisting}

%--
