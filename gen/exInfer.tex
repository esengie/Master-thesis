\subsubsection{Пример генерации функции вывода типов}
Итак, на данной стадии работы алгоритма у нас имеется ассоциативный массив сопоставляющий метаперенные с переменными Haskell, которые им соответствуют. Рассмотрим дальнейший ход действий на примере. Предположим нам дано искусственное правило вывода ff (см. вставку~\ref{ffrulee}).

\begin{lstlisting}[label={ffrulee}, caption={Искусственное правило вывода для конструкции ff},captionpos=b, frame=single, float, floatplacement=H]
FRule =
    forall S : ty, t : tm, T : ty
      x:S, y:S |- t : T, -- (1)
      x:T |- t : bool,   -- (2)
      |- gf(S, (x z).rf(T, (y r).T)) : rf(S, (x z).T) -- (3)
      |-----------------
      |- ff(S, t) def
\end{lstlisting}

На момент вызова у нас есть S и t в общем для всех контексте ctx. Предположим, мы уже отсортировали предпосылки по трем группам описанным в~\ref{repr}. Сперва нам необходимо сгенерировать проверку предпосылок вводящих метапеременные, так как эти метапеременные могут быть использованы в других предпосылках (они не могут быть использованы в предпосылках, задающие другие метапеременные, так как это ограничение нашего языка спецификации, см. Раздел~\ref{constraints}, Пункт~\ref{right::}). А после этого уже проверяем остальные предпосылки.

(Ниже описан ход работы одного шага функции `infer' для правила ff и того, как работает генерация её кода. Дальнейший текст лучше читать имея вставку~\ref{FRule} перед глазами.)

Поэтому первой предпосылкой которую мы проверяем является `\lstinline{x:S, y:S |- t : T}'. Чтобы получить выражение соответствующее (x y).T, мы должны вызвать функцию вывода типов в контексте, который нам передан, расширенном двумя вхождениями типа S. Для этого мы должны расширить контекст выражения t. Это является ограничением, наложенным на нас нашим же представлением, так как иначе у нас не сойдутся типы.

Правда мы должны сперва проверить, что эти расширения контекста определены, то есть проверять определенность каждого типа, который мы добавляем в контекст -- в данном случае это выльется в два вызова функции проверки типа и один вызов функции `infer', именно это и происходит в `check (1)' во вставке~\ref{FRule}. `rt add1' просто добавляет связывание между выражением и переменными контекста как описано в Разделе~\ref{build_exp}.

Затем, получив переменную `v3', соответствующую нашей метапеременной в увеличенном контексте $\Gamma, x, y \vdash T$, до того как мы добавим её в таблицу метапеременных кодогенерации, мы должны уменьшить её контекст до того контекста, что указан в forall, а если это невозможно --- то оповестить об ошибке проверки типов. Что и происходит в соответствующей строке кода.
% Мы должны сперва привести выражение к нормальной форме и только потом пытаться удалять переменные, так как у нас эквивалентность по нормальной форме

После проверки первой мы проверяем вторую предпослыку, аналогично описанному выше способу. В третей предпосылке мы должны построить выражение `\lstinline{gf(S, (x z).rf(T, (y r).T))}'. Это делается рекурсивно внутри монады кодогенерации, чтобы мы имели доступ к нашему ассоциативному массиву метапеременных. В данном примере нам потребуется из `x.T' получить `(x z).T' и `(x z y r).T', что мы и делаем.

После этого все аналогично первому примеру, кроме последней неоговоренной детали --- при получении типа мы должны проверить его на равенство типу  `\lstinline{rf(S, (x z).T)}'. Который мы строим аналогично предыдущему описанию, затем вызываем функцию проверки равенства типов, на типе полученном при выводе типа `\lstinline{gf(S, (x z).rf(T, (y r).T))}' и построенном из метапеременных типа `\lstinline{rf(S, (x z).T)}' (см. вставку~\ref{FRule}, пункт под названием `equality check'). В конце работы `infer' мы выдаем тип нашей конструкции.

\begin{lstlisting}[caption={Искусственный пример случая несоответствия контекстов: контекст t нужно сократить до использования в предпосылке.},label={FRule},captionpos=b, frame=single, float, floatplacement=H]
 rt f x = runIdentity (traverse f x)
 infer :: (Show a, Eq a) => Ctx a -> Term a -> TC (Type a)
 infer ctx (Ff v1 v2) = do

     -- check (1) -----------------------------------------

     checkT ctx TyDef v1
     checkT (consCtx v1 ctx) TyDef (rt add1 v1)
     v3 <- infer (consCtx (rt add1 v1) (consCtx v1 ctx))
             (rt add1 (rt add1 v2))

     -- (x y).T --> T -----------------------------------------

     v4 <- pure (nf v3) >>= traverse rem1 >>= traverse rem1

     -- check (2) -----------------------------------------

     checkT ctx TyDef v4
     v6 <- infer (consCtx v4 ctx) (rt add1 v2)
     checkEq Bool v6

     -- check (3) -----------------------------------------

     v5 <- infer ctx
             (Gf v1
                (toScope2
                   (Rf (rt add1 (rt add1 v4))
                      (toScope2 (rt add1 (rt add1 (rt add1 (rt add1 v4))))))))

     -- equality check -----------

     checkEq (Rf v1 (toScope2 (rt add1 (rt add1 v4)))) v5

     -- return --------------------------------------------

     pure TyDef
\end{lstlisting}
