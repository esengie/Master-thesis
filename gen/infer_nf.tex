\pagebreak
\subsection{Вывод типов и нормализация}\label{nf_infer}
Сам `infer' работает как описано в Разделе~\ref{typecheck}. Мы последовательно строим каждую предпосылку и вызываем функцию вывода типов или проверки типа выражения на равенство типу. Выражения, которые мы передаем в эти функции, строим последовательно, на основе переданных нам в конструкции в заключении, или полученных при вызове функции вывода типов (подробнее об этом ниже).

Внутри кодогенерации функций `nf' и `infer' мы должны уметь строить выражения языка, который мы проверяем/редуцируем. В функции infer это связано с тем, что при проверке предпосылок и возврате типа конструкции мы должны уметь строить произвольные выражения специфицированного языка. В функции `nf' это связано с тем, что правая часть редукции может содержать произвольные выражения языка.

\subsubsection{Пример генерации функции вывода типов}
Итак, на данной стадии работы алгоритма у нас имеется ассоциативный массив сопоставляющий метаперенные с переменными Haskell, которые им соответствуют. Рассмотрим дальнейший ход действий на примере. Предположим нам дано искусственное правило вывода `ff' (см. вставку~\ref{ffrulee}).

\begin{lstlisting}[label={ffrulee}, caption={Искусственное правило вывода для конструкции `ff'},captionpos=b, frame=single, float, floatplacement=H]
FRule =
    forall S : ty, t : tm, T : ty
      x:S, y:S |- t : T, -- (1)
      x:T |- t : bool,   -- (2)
      |- gf(S, (x z).rf(T, (y r).T)) : rf(S, (x z).T) -- (3)
      |-----------------
      |- ff(S, t) def
\end{lstlisting}

На момент вызова у нас есть выражения S и t в общем для всех контексте ctx. Предположим, мы уже отсортировали предпосылки по трем группам описанным в~\ref{repr}. Сперва нам необходимо сгенерировать проверку предпосылок вводящих метапеременные, так как эти метапеременные могут быть использованы в других предпосылках (они не могут быть использованы в предпосылках, задающие другие метапеременные, так как это ограничение нашего языка спецификации, см. Раздел~\ref{constraints}, Пункт~\ref{right::}). А после этого уже проверяем остальные предпосылки.

(Ниже описан ход работы одного шага функции infer для правила ff и того, как работает генерация её кода. Дальнейший текст лучше читать имея вставку~\ref{FRule} перед глазами.)

Поэтому первой предпосылкой которую мы проверяем является `\lstinline{x:S, y:S |- t : T}'. Чтобы получить выражение соответствующее (x y).T, мы должны вызвать функцию вывода типов в контексте, который нам передан, расширенном двумя вхождениями типа S. Для этого мы должны расширить контекст выражения t. Это является ограничением, наложенным на нас нашим же представлением, так как иначе у нас не сойдутся типы.

Правда мы должны сперва проверить, что эти расширения контекста определены, то есть проверять определенность каждого типа, который мы добавляем в контекст -- в данном случае это выльется в два вызова функции проверки типа и один вызов функции infer, именно это и происходит в `check (1)' во вставке~\ref{FRule}. `rt add1' просто добавляет связывание между выражением и переменными контекста как описано в Разделе~\ref{build_exp}.

Затем, получив переменную v3, соответствующую нашей метапеременной в увеличенном контексте `$\Gamma, x, y \vdash T$', до того как мы добавим её в таблицу метапеременных кодогенерации, мы должны уменьшить её контекст до того контекста, что указан в forall, а если это невозможно --- то оповестить об ошибке проверки типов. Что и происходит в соответствующей строке кода.
% Мы должны сперва привести выражение к нормальной форме и только потом пытаться удалять переменные, так как у нас эквивалентность по нормальной форме

После проверки первой мы проверяем вторую предпослыку, аналогично описанному выше способу. В третей предпосылке мы должны построить выражение `\lstinline{gf(S, (x z).rf(T, (y r).T))}'. Это делается рекурсивно внутри монады кодогенерации, чтобы мы имели доступ к нашему ассоциативному массиву метапеременных. В данном примере нам потребуется из `x.T' получить `(x z).T' и `(x z y r).T', что мы и делаем.

После этого все аналогично первому примеру, кроме последней неоговоренной детали --- при получении типа мы должны проверить его на равенство типу  `\lstinline{rf(S, (x z).T)}'. Который мы строим аналогично предыдущему описанию, затем вызываем функцию проверки равенства типов, на типе полученном при выводе типа `\lstinline{gf(S, (x z).rf(T, (y r).T))}' и построенном из метапеременных типа `\lstinline{rf(S, (x z).T)}' (см. вставку~\ref{FRule}, пункт под названием `equality check'). В конце работы infer мы выдаем тип нашей конструкции.

\begin{lstlisting}[caption={Искусственный пример случая несоответствия контекстов: контекст t нужно сократить до использования в предпосылке.},label={FRule},captionpos=b, frame=single, float, floatplacement=H]
 rt f x = runIdentity (traverse f x)
 infer :: (Show a, Eq a) => Ctx a -> Term a -> TC (Type a)
 infer ctx (Ff v1 v2) = do

     -- check (1) -----------------------------------------

     checkT ctx TyDef v1
     checkT (consCtx v1 ctx) TyDef (rt add1 v1)
     v3 <- infer (consCtx (rt add1 v1) (consCtx v1 ctx))
             (rt add1 (rt add1 v2))

     -- (x y).T --> T -----------------------------------------

     v4 <- pure (nf v3) >>= traverse rem1 >>= traverse rem1

     -- check (2) -----------------------------------------

     checkT ctx TyDef v4
     v6 <- infer (consCtx v4 ctx) (rt add1 v2)
     checkEq Bool v6

     -- check (3) -----------------------------------------

     v5 <- infer ctx
             (Gf v1
                (toScope2
                   (Rf (rt add1 (rt add1 v4))
                      (toScope2 (rt add1 (rt add1 (rt add1 (rt add1 v4))))))))

     -- equality check -----------

     checkEq (Rf v1 (toScope2 (rt add1 (rt add1 v4)))) v5

     -- return --------------------------------------------

     pure TyDef
\end{lstlisting}

\pagebreak
\subsubsection{Пример генерации функции нормализации}

Функция nf пытается сопоставиться с образцом на выражении, если это не выходит, то данная редукция неприменима. Поэтому нужен способ отслеживать, какие редукции уже были опробованы, а какие нет. Это делается с помощью структуры данных Cnt:

\begin{lstlisting}
data Cnt = Bot | U (Cnt)
  deriving(Eq, Show)
\end{lstlisting}

Эта структура служит в качестве целого числа, которое мы инициализируем количеством редукций применимых к нашей кострукции языка, и каждая несовпавшая редукция вызывает функцию nf c меньшим и меньшим числом, если же мы доходим до нуля (Bot), то мы истощили набор применимых редукций --- а значит выражение находится в нормальной форме.

Стоит отметить, что такое сопоставление с образцом невозможно с использованием библиотеки bound\cite{bound}, поэтому был написан модуль SimpleBound c обычными, а не обобщенными, индексами де Брейна (это было упомянуто в Разделе~\ref{de_brujin_impl}).

Рассмотрим работу алгоритма на примере редукций конструкции `If' (см. вставку~\ref{iffreds}).

\begin{lstlisting}[label={iffreds}, caption={Правила редукций для конструкции If},captionpos=b, frame=single, float, floatplacement=H]
IfRed1 =
  forall x.A : ty, f : tm , g : tm
    |--- |- if(x.A, true, f, g) => f : A[x:=true]
IfRed2 =
  forall x.A : ty, f : tm , g : tm
    |--- |- if(x.A, false, f, g) => g : A[x:=true]
\end{lstlisting}

(Дальнейший текст лучше читать имея вставку~\ref{lst_if_nf} перед глазами.)

Если у конструкции есть редукции, то функция nf вызывает функцию nf', в которую передает количество возможных редукций для данной конструкции. Также она передает нормализованные внутренние выражения для того, чтобы сопоставление с образцом было корректным (так как в работе алгоритма и при описании редукций языка подразумевается нормальная форма внутренних конструкций).

Строки 6 и 15 соответствуют сопоставлению с образцом левой части редукции. Случай, когда выражение находится в нормальной форме учтен на строке 24.

Внутри каждой функции происходит попытка построения правой части редукции таким же способом, как это происходит в функции вывода типов infer. То есть так же внутри монады `TC'. Но теперь в случае ошибки мы просто пытаемся применить следующую редукцию к выражению.

Это можно заметить на строках 13 и 22 --- в случае ошибки рекурсивно вызывается функция nf' с понижением счетчика на единицу.

В случае применимости редукции мы вызываем функцию nf на правой части редукции. Стоит отметить отличие возвращаемого значения строк 14 и 23 от возвращаемого значения в строке 24. Если нет применимых редукций, то выражение в нормальной форме. Если есть, то правая часть редукции может быть редуцирована дальше.

\begin{lstlisting}[caption={Приведение в нормальную форму пытается применить все редукции данного функционального символа},captionpos=b, frame=single, float, floatplacement=H, label={lst_if_nf}, numbers=left]
nf :: (Show a, Eq a) => Term a -> Term a
nf (If v1 v2 v3 v4)
  = nf' (U (U Bot)) (If (nf1 v1) (nf v2) (nf v3) (nf v4))

nf':: (Show a, Eq a) => Cnt -> Term a -> Term a
nf' (U (U _)) al@(If (Scope v1) True v2 v3)
  = case
      do v4 <- pure v1
         v5 <- pure v2
         v6 <- pure v3
         pure v5
      of
        Left _ -> nf' (U Bot) al
        Right x -> nf x
nf' (U _) al@(If (Scope v1) False v2 v3)
  = case
      do v4 <- pure v1
         v5 <- pure v2
         v6 <- pure v3
         pure v6
      of
        Left _ -> nf' Bot al
        Right x -> nf x
nf' _ x = x
\end{lstlisting}

Во вставке~\ref{lst_lam_nf} показан пример того, что происходит, если  у конструкции нет редукций --- нормализуются выражения внутри и возвращается модифицированное выражение.

\begin{lstlisting}[caption={Приведение в нормальную форму конструкции, у которой нет редукций}, captionpos=b, frame=single, float, floatplacement=H, label={lst_lam_nf}]
nf (Lam v1 v2) = Lam (nf v1) (nf1 v2)

nf1 x = (toScope $ nf $ fromScope x)
\end{lstlisting}

