\subsection{Вывод типов и нормализация}\label{nf_infer}
Сам infer работает как описано в Разделе~\ref{typecheck}. Мы последовательно строим каждую предпосылку и вызываем функцию вывода типов или проверки типа выражения на равенство типу. Термы, которые мы передаем в эти функции, строим последовательно, на основе переданных нам в конструкции в заключении, или полученных при вызове функции вывода типов (подробнее об этом ниже).


Внутри кодогенерации функций nf и infer мы должны уметь строить термы языка, который мы проверяем/редуцируем. В функции infer это связано с тем, что при проверке предпосылок и возврате типа конструкции мы должны уметь строить произвольные термы специфицированного языка. В функции nf это связано с тем, что правая часть редукции может содержать произвольные термы языка.

Итак, на данной стадии работы алгоритма у нас имеется ассоциативный массив сопоставляющий метаперенные с выражениями на Haskell, которые им соответствуют
% (инвариант кода: это всегда просто переменные языка)
. Рассмотрим дальнейший ход действий на примере. Предположим нам дано искусственное правило вывода ff (см. вставку~\ref{ffrulee}).

\begin{lstlisting}[label={ffrulee}, caption={Искусственное правило вывода для конструкции ff},captionpos=b, frame=single, float, floatplacement=H]
FRule =
    forall S : ty, t : tm, T : ty
      x:S, y:S |- t : T, -- (1)
      x:T |- t : bool,   -- (2)
      |- gf(S, (x z).rf(T, (y r).T)) : rf(S, (x z).T) -- (3)
      |-----------------
      |- ff(S, t) def
\end{lstlisting}

На момент вызова у нас есть S и t в общем для всех контексте. Мы уже отсортировали предпосылки по трем группам описанным в~\ref{repr}. Сперва нам нужны предпослыки вводящие метапеременные, так как они могут быть использованы в других предпосылках (они не могут быть использованы в предпосылках, задающие другие метапеременные, так как это ограничение нашего языка спецификации, см. Подраздел~\ref{constraints}, Пункт~\ref{right::}).

Поэтому первой предпосылкой которую мы проверяем является `\lstinline{x:S, y:S |- t : T}'. Чтобы получить терм T, мы должны вызвать функцию вывода типов в контексте, который нам передан, расширенном двумя вхождениями типа S. Для этого мы должны расширить контекст терма t. Это является ограничением, наложенным на нас нашим же представлением, так как иначе у нас не сойдутся типы.

Правда мы должны сперва проверить, что эти расширения контекста определены, то есть проверять определенность каждого типа, который мы добавляем в контекст -- в данном случае это выльется в два вызова функции проверки типа и один вызов функции `infer', именно это и происходит в `check (1)' во вставке~\ref{FRule}. `rt add1' просто добавляет связывание между термом и переменными контекста как описано в Разделе~\ref{build_exp}.

Затем, получив нашу переменную в увеличенном контексте (в forall она имеет контекст меньшей длины), до того как мы добавим переменную, которая является её представителем в коде Haskell, мы должны уменьшить её контекст до того, что указан в forall, если это невозможно --- то оповестить об ошибке проверки типов.

Затем мы проверяем вторую предпослыку, аналогично описанному выше способу. В третей предпосылке мы должны строить терм \lstinline{gf(S, (x z).rf(T, (y r).T))}. Это делается рекурсивно внутри монады кодогенерации, чтобы мы имели доступ к нашему ассоциативному массиву метапеременных. В данном примере нам потребуется из x.T получить (x z).T и (x z y r).T. Затем все аналогично.

Но при получении типа мы должны проверить его на равенство типу  \lstinline{rf(S, (x z).T)}. Который мы строим аналогично предыдущему описанию, затем вызываем функцию проверки равенства типов, на терме полученном при выводе типа \lstinline{gf(S, (x z).rf(T, (y r).T))} и построенном из метапеременных терма \lstinline{rf(S, (x z).T)} (см. вставку~\ref{FRule}).


\begin{lstlisting}[caption={Искусственный пример случая несоответствия контекстов: контекст t нужно сократить до использования в предпосылке.},label={FRule},captionpos=b, frame=single, float, floatplacement=H]
 rt f x = runIdentity (traverse f x)
 infer :: (Show a, Eq a) => Ctx a -> Term a -> TC (Type a)
 infer ctx (Ff v1 v2) = do
     -- check (1) -----------------------------------------
     checkT ctx TyDef v1
     checkT (consCtx v1 ctx) TyDef (rt add1 v1)
     v3 <- infer (consCtx (rt add1 v1) (consCtx v1 ctx))
             (rt add1 (rt add1 v2))
     -- check (3) -----------------------------------------
     v4 <- pure (nf v3) >>= traverse rem1 >>= traverse rem1
     v5 <- infer ctx
             (Gf v1
                (toScope2
                   (Rf (rt add1 (rt add1 v4))
                      (toScope2 (rt add1 (rt add1 (rt add1 (rt add1 v4))))))))
     checkEq (Rf v1 (toScope2 (rt add1 (rt add1 v4)))) v5
     -- check (2) -----------------------------------------
     checkT ctx TyDef v4
     v6 <- infer (consCtx v4 ctx) (rt add1 v2)
     checkEq Bool v6
     -- return --------------------------------------------
     pure TyDef
\end{lstlisting}


\hfill

% \begin{lstlisting}[caption={Пример правила вывода и части сгенерированной функции infer, соответствующей этому правилу},captionpos=b, frame=single, float, floatplacement=H]
% If-then =
%     forall t : tm, t1 : tm, t2 : tm, x.A : ty
%       x : bool |- A def,   -- (1)
%       |- t1 : A[x:=true],  -- (2)
%       |- t2 : A[x:=false], -- (3)
%       |- t : bool          -- (4)
%       |-------------------------------
%       |- if(x.A, t, t1, t2) : A[x:=t]
%
% infer ctx (If v1 v2 v3 v4) = do
%      -- check (4) ------------------------------------------
%      v5 <- infer ctx v2
%      checkEq Bool v5
%      -- check (3) ------------------------------------------
%      v6 <- infer ctx v4
%      checkEq (instantiate False (toScope (fromScope v1))) v6
%      -- check (2) ------------------------------------------
%      v7 <- infer ctx v3
%      checkEq (instantiate True (toScope (fromScope v1))) v7
%      -- check (1) ------------------------------------------
%      checkT ctx TyDef Bool
%      checkT (consCtx Bool ctx) TyDef (fromScope v1)
%      -- return ---------------------------------------------
%      pure (instantiate v2 (toScope (fromScope v1)))
% \end{lstlisting}

Стоит отметить, что порядок или количество переменных метапеременных, которые нам переданы, могут отличаться от порядка и вида контекста, в котором наша метапеременная должна быть в момент её использования. Эту проблему мы решаем приводя метапеременные к контексту данному в подразделе forall нашего правила вывода методом, описанным в Разделе~\ref{build_exp}.

Функция nf пытается сопоставиться с образцом на терме, если это не выходит, то данная редукция неприменима. Поэтому нужен способ отслеживать какие редукции уже были опробованы, а какие нет. Это делается с помощью специальной структуры данных Cnt, описанной во вставке~\ref{lst_nf}. Эта структура служит в качестве целого числа, которое мы инициализируем количеством редукций применимых к нашей кострукции языка, и каждая несовпавшая редукция вызывает функцию nf c меньшим и меньшим числом, если же мы доходим до нуля, то мы истощили набор применимых редукций --- а значит терм находится в нормальной форме.

Стоит отметить, что такое сопоставление с образцом невозможно с использованием библиотеки bound\cite{bound}, поэтому был написан модуль SimpleBound c обычными, а не обобщенными, индексами де Брейна.

\begin{lstlisting}[caption={Приведение в нормальную форму пытается применить все редукции данного функционального символа},captionpos=b, frame=single, float, floatplacement=H, label={lst_nf}]
data Cnt = Bot | U (Cnt)
  deriving(Eq, Show)

nf :: (Show a, Eq a) => Term a -> Term a
nf (If v1 v2 v3 v4)
  = nf' (U (U Bot)) (If (nf1 v1) (nf v2) (nf v3) (nf v4))

nf':: (Show a, Eq a) => Cnt -> Term a -> Term a
nf' (U (U _)) al@(If (Scope v1) True v2 v3)
  = case
      do v4 <- pure v1
         v5 <- pure v2
         v6 <- pure v3
         pure v5
      of
        Left _ -> nf' (U Bot) al
        Right x -> nf x
nf' (U _) al@(If (Scope v1) False v2 v3)
  = case
      do v4 <- pure v1
         v5 <- pure v2
         v6 <- pure v3
         pure v6
      of
        Left _ -> nf' Bot al
        Right x -> nf x
nf' _ x = x
\end{lstlisting}
