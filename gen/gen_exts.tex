\subsection{Генерация кода}
Генерация кода происходит с использованием библиотеки haskell-src-exts\cite{src_exts}, которая дает нам функции генерации и манипуляции АСД Haskell.

В виду того, что мы выбрали представление в виде Индексов де Брейна с полиморфной рекурсией, большинство кода для работы с представлением языка и контекстами не зависит от самого языка. От нас требуется только генерация определений четрырёх сущностей:

\begin{itemize}
\item Представления специфицированного языка
\item Представителя класса Monad для представления языка
\item Функции infer --- вывода типов терма
\item Функции nf --- приведения в нормальную форму/вычисления терма
\end{itemize}

Поэтому, мы просто модифицируем написанный от руки модуль LangTemplate, в котором уже написаны функции работы с контекстами, функции проверки типов (см. вставку~\ref{lst_checkT}) и проверки термов на равенство. Также написаны заглушки для четырех сущностей, указанных выше.

Проверка типов специфицированного языка происходит внутри монады TC, определенной как `Either Left', чтобы можно было сообщать пользователю об ошибках.

Например, так работает функция проверки типов: в неё передается контекст, тип и терм; функция вызывает `infer' от контекста и терма и затем сравнивает нормальные формы типа, переданного в качестве аргумента, и типа, полученного при помощи вызова функции `infer'.

Функция увеличения контекста `consCtx' используется при проверке типов, а значит должна увеличивать индексы де Брейна контекста по мере продвижения вглубь терма. Это связано с тем, что самые новые переменные могут зависеть от более старых переменных контекста. Увеличиваются индексы с помощью fmap, чтобы увеличивались контексты внутри типов, которые хранятся в контексте. Интерес представляет возвращаемый тип, но это всего лишь означает, что функция теперь может возвращать типы переменных с большими индексами де Брейна.

Так же работает рекурсивный вызов с заходом под связывание терма в функциях `infer' или `nf' --- увеличение контекста отражается на типах.

\begin{lstlisting}[caption={Проверка типов и контексты},captionpos=b, frame=single, float,floatplacement=H, label={lst_checkT}]
type TC    = Either String
type Ctx a = a -> TC (Type a)

emptyCtx :: (Show a, Eq a) => Ctx a
emptyCtx x = Left $ "Variable not in scope: " ++ show x

consCtx :: (Show a, Eq a) => Type a -> Ctx a -> Ctx (Var a)
consCtx ty ctx B = pure (F <$> ty)
consCtx ty ctx (F a)  = (F <$>) <$> ctx a

checkT :: (Show a, Eq a) => Ctx a -> Type a -> Term a -> TC ()
checkT ctx want t = do
  have <- infer ctx t
  when (nf have /= nf want) $ Left $
    "type mismatch, have: " ++ (show have) ++ " want: " ++ (show want)
\end{lstlisting}

Всё остальное генерируется с помощью Template Haskell\cite{TH} --- представители классов Traversable\cite{deriveFun}, Functor, Foldable (Foldable дает нам функцию toList, которая возвращает свободные переменные терма, Traversable позволяет применять функции swap, rem и add к переменным внутри терма). Также представители классов Eq и Show (описано в Подразделе~\ref{final_repr}).


%%
